%% -*- Mode: LaTeX -*-
%%
%% branching-tagging.tex
%% Created Tue Nov 17 15:18:14 AKST 2015
%% Copyright (C) 2015 by Raymond E. Marcil <marcilr@gmail.com>
%%
%% Branching & Tagging
%%

%% =================== Branching & Tagging =======================
%% =================== Branching & Tagging =======================
\newpage
\section{Branching \& Tagging}

In short: Best practice is branch out, merge often and keep
always in sync.\\
\\
\noindent There are pretty clear conventions about keeping your
code in a separate branches from master branch:

\begin{enumerate*}
\item You are about to make an implementation of major or
disruptive change
\item You are about to make some changes that might not be used
\item You want to experiment on something that you are not sure
    it will work
\item When you are told to branch out, others might have
    something they need to do in master
\end{enumerate*}

\noindent Rule of thumb is after branching out, you should keep in sync with
the master branch. Because eventually you need to merge it back to master. In
order to avoid a huge complicated mess of conflicts when merging back, you
should commit often, merge often.\footnote{Git branching and tagging best
practices\\
\href{http://programmers.stackexchange.com/questions/165725/git-branching-and-tagging-best-practices}{http://programmers.stackexchange.com/questions/165725/git-branching-and-tagging-best-practices}}
