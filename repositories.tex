%% -*- Mode: LaTeX -*-
%%
%% remote-repositories.tex
%% Created Wed Nov 18 15:00:41 AKST 2015
%% Copyright (C) 2015 by Raymond E. Marcil <marcilr@gmail.com>
%%
%% Remote Repositories
%%


%% ================== Remote Repositories ========================
%% ================== Remote Repositories ========================
\subsubsection{Remote Repositories}
GitHub's collaborative approach to development depends on
publishing commits from your local repository for other people to
view, fetch, and update.\footnote{About remote repositories,
\href{https://help.github.com/articles/about-remote-repositories/}{https://help.github.com/articles/about-remote-repositories/}}
\\
\\
A remote URL is Git's fancy way of saying ``the place where your
code is stored.'' That URL could be your repository on GitHub, or
another user's fork, or even on a completely different server.
\\
\\
You can only push to two types of URL addresses:
\\
\\
$>$ An HTTPS URL like \cmd{https://github.com/user/repo.git}\\
$>$ An SSH URL, like \cmd{git@github.com:user/repo.git}\\
\\
Git associates a remote URL with a name, and your default remote is usually called origin.
