%% -*- Mode: LaTeX -*-
%%
%% init.tex
%% Created Fri Nov 20 11:16:34 AKST 2015
%% Copyright (C) 2015 by Raymond E. Marcil <marcilr@gmail.com>
%%
%% Google repo init
%%

%% ========================== init ===============================
%% ========================== init ===============================
\clearpage
\subsubsection{init}
\FIXME{Good init syntax here:\\
\href{https://source.tizen.org/documentation/developer-guide/getting-started-guide/cloning-tizen-source}{https://source.tizen.org/documentation/developer-guide/getting-started-guide/cloning-tizen-source}}
\\
\\
\cmd{\$ repo init -u <URL> [<OPTIONS>]}
\\
\\
Installs Repo in the current directory. This creates a \cmd{.repo/}
directory that contains Git repositories for the Repo source code
and the standard Android manifest files.  The \cmd{.repo/}
directory also contains \cmd{manifest.xml}, which is a symlink to the
selected manifest in the \cmd{.repo/manifests/} directory.\footnote{Repo command reference --
\href{https://source.android.com/source/using-repo.html\#forall}{https://source.android.com/source/using-repo.html}}\\

\begin{table}[htb]
\begin{center}
\begin{tabular}{|p{.15\textwidth}|p{.75\textwidth}|}\hline
\centering Command&\centering Description\tabularnewline
\hline
\centering -u&Specify a URL from which to retrieve a manifest repository.
The common manifest can be found at:
\cmd{https://android.googlesource.com/platform/manifest}\\
\centering -m&Select a manifest file within the repository. If no manifest
 name is selected, the default is default.xml.\\
\centering -b&Specify a revision, i.e., a particular manifest-branch.\\
\hline
\end{tabular}
\caption {init Options}
\label{table:init_options}
\end{center}
\end{table}

%% ======================== Examples =============================
%% ======================== Examples =============================
\noindent \begin{bf}Examples\end{bf}
\\
\\
This will create a new place to hold your local copy of the source
tree.  The \cmd{url} should point to a Manifest repository that
describes the whole sources.  It is a special project with a file
(default.xml) that lists all the projects that Android is made of.
In the Manifest file, each projects has attributes about: where to
place it in the tree, where to download it from (git server),
revision that will be used (usually a branch name, tag or commit
sha-id).\footnote{Repo: Tips \& Tricks,\\
\href{http://xda-university.com/as-a-developer/repo-tips-tricks}{http://xda-university.com/as-a-developer/repo-tips-tricks}}
\\
\\
\indent\cmd{repo init -u <url> -b <branch>}
\\
\\
\noindent Note: For all remaining Repo commands, the current working
directory must either be the parent directory of \cmd{.repo/}
or a subdirectory of the parent directory.\footnote{Repo command reference --
\href{https://source.android.com/source/using-repo.html\#forall}{https://source.android.com/source/using-repo.html}}\\
\\
\\
\noindent \FIXME{Need example of GitHub checkout}
