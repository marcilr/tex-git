%% -*- Mode: LaTeX -*-
%%
%% ssh.tex
%% Created Wed Nov 18 15:00:41 AKST 2015
%% Copyright (C) 2015 by Raymond E. Marcil <marcilr@gmail.com>
%%
%% The SSH Protocol
%%


%% ==================== The SSH Protocol =========================
%% ==================== The SSH Protocol =========================
\newpage
\subsubsection{SSH}
A common transport protocol for Git when self-hosting is over SSH.
This is because SSH access to servers is already set up in most
places – and if it isn’t, it's easy to do.  SSH is also an
authenticated network protocol; and because it’s ubiquitous,
it's generally easy to set up and use.
\\
\\
If you have two-factor authentication\footnote{About Two-Factor Authentication,
\href{https://help.github.com/articles/about-two-factor-authentication/}{https://help.github.com/articles/about-two-factor-authentication/}}
enabled, you must create a personal access token\footnote{Creating
an access token for command-line use to use instead of your GitHub password.,
\href{https://help.github.com/articles/creating-an-access-token-for-command-line-use/}{https://help.github.com/articles/creating-an-access-token-for-command-line-use/}}
 to use instead of your GitHub password.\footnote{Changing a remote's URL,
 \href{https://help.github.com/articles/changing-a-remote-s-url/}{https://help.github.com/articles/changing-a-remote-s-url/}}
\\
\\
To clone a Git repository over SSH, you can specify ssh:// URL like this:

\begin{Verbatim}
 $ git clone ssh://user@server/project.git
\end{Verbatim}

\noindent Or you can use the shorter scp-like syntax for the SSH
protocol:

\begin{Verbatim}
 $ git clone user@server:project.git
\end{Verbatim}

\noindent You can also not specify a user, and Git assumes the
user you’re currently logged in as.


%% ======================== The Pros =============================
%% ======================== The Pros =============================
\vspace{20pt}
\noindent\begin{bf}The Pros\end{bf}
The pros of using SSH are many.  First, SSH is relatively easy
to set up – SSH daemons are commonplace, many network admins have
experience with them, and many OS distributions are set up with
them or have tools to manage them.  Next, access over SSH is
secure – all data transfer is encrypted and authenticated.  Last,
like the HTTP/S, Git and Local protocols, SSH is efficient, making
the data as compact as possible before transferring it.


%% ======================== The Cons =============================
%% ======================== The Cons =============================
\vspace{20pt}
\noindent\begin{bf}The Cons\end{bf}
The negative aspect of SSH is that you can’t serve anonymous
access of your repository over it.  People must have access to
your machine over SSH to access it, even in a read-only capacity,
which doesn’t make SSH access conducive to open source projects.
If you’re using it only within your corporate network, SSH may be
the only protocol you need to deal with.  If you want to allow
anonymous read-only access to your projects and also want to use
SSH, you’ll have to set up SSH for you to push over but something
else for others to fetch over.\footnote{Ibid.}
