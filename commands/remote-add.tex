%% -*- Mode: LaTeX -*-
%%
%% remote-add.tex
%% Created Thu Nov 19 08:25:10 AKST 2015
%% Copyright (C) 2015 by Raymond E. Marcil <marcilr@gmail.com>
%%
%% git remote add
%%


%% ===================== git remote add =========================
%% ===================== git remote add =========================
\newpage
\subsection{git remote add}
To add a new remote, use the \cmd{git remote add} command on the
terminal, in the directory your repository is stored at.\footnote{Adding a remote,
\href{https://help.github.com/articles/adding-a-remote/}{https://help.github.com/articles/adding-a-remote/}}
\\
\\
The \cmd{git remote add} command takes two arguments:
\vspace{-10pt}
\begin{verbatim}
$ git remote add <NAME> <REMOTE_URL> 
\end{verbatim}

\noindent Where:
\vspace{-10pt}
\begin{table}[htb]
%%\begin{table}
\begin{center}
\begin{tabular}{p{.20\textwidth}p{.80\textwidth}}
\hspace{20pt}\cmd{<NAME>}&- A remote name, for example, \cmd{origin}\\
\hspace{20pt}\cmd{<REMOTE\_URL>}&- A remote URL, for example, \cmd{https://github.com/user/repo.git}\\
\end{tabular}
\label{table:remote_add}
\end{center}
\end{table}

\vspace{-20pt}
\noindent Git associates a remote URL with a name, and your
default remote is usually called origin.\footnote{About remote repositories,
\href{https://help.github.com/articles/about-remote-repositories/}{https://help.github.com/articles/about-remote-repositories/}}
\\

%% ======================== Examples ============================
%% ======================== Examples ============================
\vspace{10pt}
\noindent \begin{bf}Examples\end{bf}
\begin{Verbatim}
 # This associates the name origin with SSH URL for repo.git repository.
 $ git remote add origin git@github.com:user/repo.git
\end{Verbatim}

\noindent Alternatively using https syntax:

\begin{Verbatim}
 $ git remote add origin https://github.com/user/repo.git
 # Set a new remote

 $ git remote -v
 # Verify new remote
 origin  https://github.com/user/repo.git (fetch)
 origin  https://github.com/user/repo.git (push)
\end{Verbatim}
