%% -*- Mode: LaTeX -*-
%%
%% remote-set-url.tex
%% Created Thu Nov 19 08:25:10 AKST 2015
%% Copyright (C) 2015 by Raymond E. Marcil <marcilr@gmail.com>
%%
%% git remote set url
%%


%% =================== git remote set-url =======================
%% =================== git remote set-url =======================
\newpage
\subsubsection{Remote set-url}
The \cmd{git remote set-url} command changes an existing remote
repository URL.\footnote{Changing a remote's URL,
\href{https://help.github.com/articles/changing-a-remote-s-url/}{https://help.github.com/articles/changing-a-remote-s-url/}}
\\
\\
The git remote set-url command takes two arguments:
\\
$>$ An existing remote name. For example, \cmd{origin} or \cmd{upstream} are two common choices.\\
$>$ A new URL for the remote. For example:\\

\noindent\hspace*{10pt}$>$ If you're updating to use HTTPS, your URL might look like:\\
\noindent\hspace*{23pt}\cmd{https://github.com/USERNAME/OTHERREPOSITORY.git}\\
\\
\noindent\hspace*{10pt}$>$ If you're updating to use SSH, your URL might look like:\\
\noindent\hspace*{23pt}\cmd{git@github.com:USERNAME/OTHERREPOSITORY.git}


%% ========================== Examples =========================
%% ========================== Examples =========================
\vspace{20pt}
\noindent\begin{bf}Examples\end{bf}
\\

%% ========= Switching remote URLs from SSH to HTTPS ===========
%% ========= Switching remote URLs from SSH to HTTPS ===========
\noindent\begin{bf}Switching remote URLs from SSH to HTTPS\end{bf}
\begin{enumerate}
  \item{Open Terminal (for Mac and Linux users) or the command prompt
       (for Windows users).\footnote{Switching remote URLs from HTTPS to SSH,
       \href{https://help.github.com/articles/changing-a-remote-s-url/}{https://help.github.com/articles/changing-a-remote-s-url/}}}
  \item{Change the current working directory to your local project.}
  \item{List your existing remotes in order to get the name of the remote you want to change.
\begin{Verbatim}
 $ git remote -v
 # origin  git@github.com:USERNAME/REPOSITORY.git (fetch)
 # origin  git@github.com:USERNAME/REPOSITORY.git (push)
\end{Verbatim}
       }
  \item{Change your remote's URL from SSH to HTTPS with the git remote set-url command.
\begin{Verbatim}
 $ git remote set-url origin \
 https://github.com/USERNAME/OTHERREPOSITORY.git
\end{Verbatim}
      }

  \item{Verify that the remote URL has changed.
\begin{Verbatim}
$ git remote -v
# Verify new remote URL
# origin  https://github.com/USERNAME/OTHERREPOSITORY.git (fetch)
# origin  https://github.com/USERNAME/OTHERREPOSITORY.git (push)
\end{Verbatim}
       }
\end{enumerate}


%% ========= Switching remote URLs from HTTPS to SSH ===========
%% ========= Switching remote URLs from HTTPS to SSH ===========
\newpage
\begin{bf}Switching remote URLs from HTTPS to SSH\end{bf}
\begin{enumerate}
  \item{Open Terminal (for Mac and Linux users) or the command prompt
       (for Windows users).\footnote{Switching remote URLs from HTTPS to SSH,
       \href{https://help.github.com/articles/changing-a-remote-s-url/}{https://help.github.com/articles/changing-a-remote-s-url/}}}
  \item{Change the current working directory to your local project.}
  \item{List your existing remotes in order to get the name of the remote you want to change.
\begin{Verbatim}
 $ git remote -v
 origin  https://github.com/USERNAME/REPOSITORY.git (fetch)
 origin  https://github.com/USERNAME/REPOSITORY.git (push)
\end{Verbatim}
       }
  \item{Change your remote's URL from HTTPS to SSH with the git remote set-url command.
\begin{Verbatim}
 $ git remote set-url origin \
 git@github.com:USERNAME/OTHERREPOSITORY.git
\end{Verbatim}
      }

  \item{Verify that the remote URL has changed.
\begin{Verbatim}
 $ git remote -v
 # Verify new remote URL
 origin  git@github.com:USERNAME/OTHERREPOSITORY.git (fetch)
 origin  git@github.com:USERNAME/OTHERREPOSITORY.git (push)
\end{Verbatim}
       }
\end{enumerate}
