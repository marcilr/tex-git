%% -*- Mode: LaTeX -*-
%%
%% init.tex
%% Created Wed Dec  2 13:50:35 AKST 2015
%% Copyright (C) 2015 by Raymond E. Marcil <marcilr@gmail.com>
%%
%% Init
%%

%% ============================ Init ==============================
%% ============================ Init ==============================
\subsubsection{Init}
Initialize new repository
\\
\\
The \cmd{git init} command creates a new Git repository.  It can be
used to convert an existing, unversioned project to a Git
repository or initialize a new empty repository. Most of the other
Git commands are not available outside of an initialized
repository, so this is usually the first command you’ll run in a
new project.
\\
\\
Executing \cmd{git init} creates a \cmd{.git} subdirectory in the
project root, which contains all of the necessary metadata for the
repo. Aside from the .git directory, an existing project remains
unaltered (unlike SVN, Git doesn't require a \cmd{.git} folder in
every subdirectory).\footnote{Setting up a repository, git init,\\
\href{https://www.atlassian.com/git/tutorials/setting-up-a-repository}
{https://www.atlassian.com/git/tutorials/setting-up-a-repository}}
\\
\\
\begin{bf}Examples\end{bf}

\begin{Verbatim}
 $ mkdir wti
 $ cd wti/
 $ git init
 Initialized empty Git repository in /home/marcilr/wti/.git/
 $ 
\end{Verbatim}

