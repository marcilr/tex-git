%% -*- Mode: LaTeX -*-
%%
%% git-remote-rm.tex
%% Created Thu Nov 19 08:25:10 AKST 2015
%% Copyright (C) 2015 by Raymond E. Marcil <marcilr@gmail.com>
%%
%% git remote rm
%%


%% ====================== git remote rm ======================
%% ====================== git remote rm ======================
\newpage
\subsubsection{git remote rm}
Use the \cmd{git remote rm} command to remove a remote URL
from your repository.\footnote{Removing a remote,
\href{https://help.github.com/articles/removing-a-remote/}{https://help.github.com/articles/removing-a-remote/}}
\\
\\
The \cmd{git remote rm} command takes one argument:
\\
\\
$>$ A remote name, for example, destination
\\
\\
\noindent \begin{bf}Example\end{bf}
\\
\\
The examples below assume you're cloning using HTTPS, which is recommended.

\begin{Verbatim}
 $ git remote -v
 # View current remotes
 origin  https://github.com/OWNER/REPOSITORY.git (fetch)
 origin  https://github.com/OWNER/REPOSITORY.git (push)
 destination  https://github.com/FORKER/REPOSITORY.git (fetch)
 destination  https://github.com/FORKER/REPOSITORY.git (push)

 $ git remote rm destination
 # Remove remote
 $ git remote -v
 # Verify it's gone
 origin  https://github.com/OWNER/REPOSITORY.git (fetch)
 origin  https://github.com/OWNER/REPOSITORY.git (push)
\end{Verbatim}

\noindent Note: \cmd{git remote rm} does not delete the remote
repository from the server.  It simply removes the remote and
its references from your local repository.
