%% -*- Mode: LaTeX -*-
%%
%% introduction.tex
%% Created Wed Nov 18 15:00:41 AKST 2015
%% Copyright (C) 2015 by Raymond E. Marcil <marcilr@gmail.com>
%%
%% Introduction
%%

%% ====================== Introduction ===========================
%% ====================== Introduction ===========================
%% ====================== Introduction ===========================
\newpage
\section{Introduction}

\FIXME{Update with great git architecture details from:\\
\href{https://git-scm.com/book/en/v2/Getting-Started-Git-Basics}{https://git-scm.com/book/en/v2/Getting-Started-Git-Basics}}
\\
\\
Git is a distributed revision control system with an emphasis on
speed,\footnote{ Torvalds, Linus
  (2005-04-07). ``\href{http://marc.info/?l=linux-kernel&m=111288700902396}{Re:
    Kernel SCM saga...}'' linux-kernel (Mailing list). ``So I'm writing some
  scripts to try to track things a whole lot faster.''} data integrity,\footnote{ Torvalds, Linus (2007-06-10). ``\href{http://marc.info/?l=git&m=118143549107708}{Re: fatal: serious inflate inconsistency}''. git (Mailing list). A brief description of Git's data integrity design goals.} and support for distributed, non-linear workflows.\footnote{Linus Torvalds (2007-05-03). \href{https://www.youtube.com/watch?v=4XpnKHJAok8}{Google tech talk: Linus Torvalds on git}. Event occurs at 02:30. Retrieved 2007-05-16.} Git was initially designed and developed by Linus Torvalds for Linux kernel development in 2005, and has since become one of the most widely adopted version control systems for software development.\footnote{ ``\href{http://ianskerrett.wordpress.com/2014/06/23/eclipse-community-survey-2014-results/}{Eclipse Community Survey 2014 results | Ian Skerrett}''. Ianskerrett.wordpress.com. 2014-06-23. Retrieved 2014-06-23.}\\
\\
\noindent As with most other distributed revision control systems,
and unlike most client–server systems, every Git working directory is a
full-fledged repository with complete history and full version-tracking
capabilities, independent of network access or a central
server.\footnote{Chacon, Scott (24
  December 2014). \href{http://git-scm.com/book/en/v2}{Pro Git} (2nd ed.). New
  York, NY: Apress. pp. 29–30. ISBN 978-1484200773.} Like the Linux kernel,
Git is free software distributed under the terms of the GNU General Public
License version 2.\footnote{Git (software), From Wikipedia, the free
encyclopedia, \href{https://en.wikipedia.org/wiki/Git\_(software)}{https://en.wikipedia.org/wiki/Git\_(software)}}

%% -*- Mode: LaTeX -*-
%%
%% workflow.tex
%% Created Fri Nov 20 09:15:29 AKST 2015
%% Copyright (C) 2015 by Raymond E. Marcil <marcilr@gmail.com>
%%
%% Workflow
%%

%% ========================= Workflow ============================
%% ========================= Workflow ============================
\subsection{Workflow}
The usual git workflow looks like:
\begin{enumerate}
  \item{Do some programming.}
  \item{\cmd{git status} to see what files I changed.}
  \item{\cmd{git diff [file]} to see exactly what I modified.}
  \item{\cmd{git commit -a -m [message]} to commit.}
\end{enumerate}

