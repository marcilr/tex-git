%% -*- Mode: LaTeX -*-
%%
%% repo.tex
%% Created Tue Nov 17 15:18:14 AKST 2015
%% Copyright (C) 2015 by Raymond E. Marcil <marcilr@gmail.com>
%%
%% Google'd repo command
%%

%% ========================== Repo ===============================
%% ========================== Repo ===============================
%% ========================== Repo ===============================
\newpage
\section{Repo}
``Repo is a repository management tool that we built on top of Git.
Repo unifies the many Git repositories when necessary, does the
uploads to a revision control system, and automates parts of the
development workflow.  Repo is not meant to replace Git, only to
make it easier to work with Git in the context of Android.  The
\cmd{repo} command is an executable Python script that you can put
anywhere in your path.  In working with source files, you will use
Repo for across-network operations.  For example, with a single
Repo command you can download files from multiple repositories
into your local working directory.''\footnote{Developing -- \href{http://source.android.com/source/developing.html}{http://source.android.com/source/developing.html}}
\\
\\
\noindent \FIXME{The above \cmd{repo} quote has been heavily modified.
Need to rewrite with original verbage.}

%% -*- Mode: LaTeX -*-
%%
%% configuration.tex
%% Created Wed Nov 18 15:00:41 AKST 2015
%% Copyright (C) 2015 by Raymond E. Marcil <marcilr@gmail.com>
%%
%% Configuration
%%

%% ===================== Configuration ===========================
%% ===================== Configuration ===========================
\subsection{Configuration}
The \cmd{repo} utility requires a \cmd{.repo} directory containing
\cmd{manifest.xml} listing repositories to manage.

%% .repo directory
%% -*- Mode: LaTeX -*-
%%
%% repo.tex
%% Created Wed Nov 18 15:00:41 AKST 2015
%% Copyright (C) 2015 by Raymond E. Marcil <marcilr@gmail.com>
%%
%% .repo directory
%%

%% ==================== .repo directory ===========================
%% ==================== .repo directory ===========================
\subsubsection{.repo directory}
The \cmd{.repo/} directory, located in the repository base,
holds repo configuration.  The configuration includes a manifest
with information about all the projects and where their associated git
repositories are located.
\\
\\
Files within the \cmd{.repo/} subdirectory includes:
\begin{verbatim}
  manifests/
  manifests.git
  manifest.xml -> manifests/default.xml
  project-objects
  projects/
  repo/
\end{verbatim}
 
\noindent To create the \cmd{.repo/} subdirectory:

\begin{Verbatim}
 $ cd <my_repo>
 $ mkdir .repo/
 $ 
\end{Verbatim}

%% Manifest
%% -*- Mode: LaTeX -*-
%%
%% manifest.tex
%% Created Fri Nov 20 11:04:59 AKST 2015
%% Copyright (C) 2015 by Raymond E. Marcil <marcilr@gmail.com>
%%
%% Manifest
%%


%% ======================== Manifest =============================
%% ======================== Manifest =============================
\clearpage
\newpage
\subsubsection{Manifest}
The repo keeps a manifest, ``within the hidden directory named `.repo','' 
in ``a git project named `\cmd{manifests}' which
usually contains a file named `\cmd{default.xml}'.  This file
contains information about all the projects and where their associated
git repositories are located.  This file is also versioned thus when
you use the `\cmd{repo init -b XYZ}' command it will be reverted and you
can back to older branches that may have added/removed git projects
compared to the head.''\footnote{How does the Android repo manifest repository work?\\
\href{http://stackoverflow.com/questions/6149725/how-does-the-android-repo-manifest-repository-work}
{http://stackoverflow.com/questions/6149725/how-does-the-android-repo-manifest-repository-work}}
\\
\\
The \cmd{default.xml} file is symlinked to \cmd{.repo/manifest.xml} and
is created when the repo was initialized using:
\\
\\
\cmd{repo init -u $<$manifest path$>$}
\\
\\

%% ======================== Examples =============================
%% ======================== Examples =============================
\noindent \begin{bf}Examples\end{bf}
\\
\\
Following is a manifest, in \cmd{.repo/manifests/default.xml} file, showing use
of GitHub with username, ssh:// URL syntax, and 3 project repos
with different usernames:\footnote{Keiji Ariyama, \href{https://github.com/keiji/repo-sample/blob/master/default.xml}{https://github.com/keiji/repo-sample/blob/master/default.xml}}

\begin{Verbatim}
 <?xml version="1.0" encoding="UTF-8"?>
 <manifest>
   <remote name="origin" fetch="ssh://git@github.com/" />
   <default revision="master" remote="origin" />

   <project path="lib/plist-lib"
            name="keiji/AndroidPListLib.git" remote="origin" />

   <project path="lib/json-pull-parser"
            name="vvakame/JsonPullParser.git" remote="origin" />

   <project path="apps/twicca_megane_plugin"
            name="zaki50/TwiccaMeganePlugin.git" remote="origin" />
 </manifest>
\end{Verbatim}



% Google Repo Commands
%% -*- Mode: LaTeX -*-
%%
%% commands.tex
%% Created Wed Nov 18 15:00:41 AKST 2015
%% Copyright (C) 2015 by Raymond E. Marcil <marcilr@gmail.com>
%%
%% Commands
%%

%% ===================== Commands =======================
%% ===================== Commands =======================
%% ===================== Commands =======================
\newpage
\subsection{Commands}

\begin{table}[htb]
\begin{center}
\begin{tabular}{|p{.15\textwidth}|p{.70\textwidth}|}\hline
Command&Description\\
\hline
\cmd{add}&Add file contents to the index\\
\cmd{apply}&Apply a patch to files and/or to the index\\
\cmd{clone}&Get a complete copy of a repository\\
\cmd{commit}&Record changes to the repository\\
\cmd{pull}&Fetch from and integrate with another repository or a local branch\\
\cmd{push}&Update remote refs along with associated objects\\
\cmd{rebase}&Forward-port local commits to the updated upstream head\\
\cmd{status}&Show the working tree status\\
\hline
\end{tabular}
\caption {Commands}
\label{table:commands}
\end{center}
\end{table}


%% Add
%% -*- Mode: LaTeX -*-
%%
%% add.tex
%% Created Wed Nov 18 15:00:41 AKST 2015
%% Copyright (C) 2015 by Raymond E. Marcil <marcilr@gmail.com>
%%
%% Add
%%

%% ============================ Add ==============================
%% ============================ Add ==============================
\subsubsection{Add}
Add file contents to the index
\\
\\
This command updates the index using the current content found in
the working tree, to prepare the content staged for the next
commit.  It typically adds the current content of existing paths
as a whole, but with some options it can also be used to add
content with only part of the changes made to the working tree
files applied, or remove paths that do not exist in the working
tree anymore.\footnote{git-add - Add file contents to the index\\
\href{https://git-scm.com/docs/git-add}{https://git-scm.com/docs/git-add}}




%% ========================= Examples ============================
%% ========================= Examples ============================
\subsubsection{Examples}
Create and add \cmd{.repo/manifests/default.xml} file and
\cmd{.repo/manifests/manifest.xml}
\begin{Verbatim}
 covellite:~/git/.repo/manifests$ git add default.xml
 covellite:~/git/.repo/manifests$ git commit default.xml
 covellite:~/git/.repo/manifests$ cd ..
 covellite:~/git/.repo/$ ln -s manifests/default.xml manifest.xml
 covellite:~/git/.repo/$ git add manifest.xml
 covellite:~/git/.repo/$ git commit
\end{Verbatim}


%% Clone
%% -*- Mode: LaTeX -*-
%%
%% clone.tex
%% Created Wed Nov 18 15:00:41 AKST 2015
%% Copyright (C) 2015 by Raymond E. Marcil <marcilr@gmail.com>
%%
%% Clone
%%

%% =========================== Clone =============================
%% =========================== Clone =============================
\newpage
\subsubsection{Clone}
To grab a complete copy of another user's repository, use git clone like this:

\begin{Verbatim}
  $ git clone https://github.com/USERNAME/REPOSITORY.git
  # Clones a repository to your computer
\end{Verbatim}

\noindent When you run \cmd{git clone}, the following actions occur:\\
$>$ A new folder called repo is made\\
$>$ It is initialized as a Git repository\\
$>$ A remote named origin is created, pointing to the URL you cloned from\\
$>$ All of the repository's files and commits are downloaded there\\
$>$ The default branch (usually called master) is checked out\\

\noindent For every branch \cmd{foo} in the remote repository, a corresponding remote-tracking
branch \cmd{refs/remotes/origin/foo} is created in your local repository.  You can
usually abbreviate such remote-tracking branch names to \cmd{origin/foo}.\footnote{Fetching a remote,
\cmd{git clone}, \cmd{git fetch}, \cmd{git merge}, \cmd{git pull},
\href{https://help.github.com/articles/fetching-a-remote/}{https://help.github.com/articles/fetching-a-remote/}}


%% ========================= Examples ============================
%% ========================= Examples ============================
\subsubsection{Examples}
To clone repository named \cmd{git} from GitHub to local \cmd{covellite} workstation:
\begin{Verbatim}
 covellite:~$ git clone https://github.com/marcilr/git.git
 Cloning into 'git'...
 warning: You appear to have cloned an empty repository.
 Checking connectivity... done.
 covellite:~$
\end{Verbatim}

\noindent Clone \cmd{.repo} repository into \cmd{git/} directory:
\begin{Verbatim}
 covellite:~/git$ git clone https://github.com/marcilr/.repo
 Cloning into '.repo'...
 warning: You appear to have cloned an empty repository.
 Checking connectivity... done.
 covellite:~/git$
\end{Verbatim}


\newpage
\noindent To clone a Git repository over SSH, you can specify ssh:// URL like this:\\
\begin{Verbatim}
 $ git clone ssh://user@server/project.git
\end{Verbatim}

\noindent Or you can use the shorter scp-like syntax for the SSH protocol:

\begin{Verbatim}
 $ git clone user@server:project.git
\end{Verbatim}

\noindent You can also not specify a user, and Git assumes the user
you're currently logged in as.\footnote{
Git on the Server - The Protocols, The SSH Protocol,\\
\href{https://git-scm.com/book/en/v2/Git-on-the-Server-The-Protocols}{https://git-scm.com/book/en/v2/Git-on-the-Server-The-Protocols}}
\\
\\
\noindent \FIXME{Need more commands here.}


%% Push
%% -*- Mode: LaTeX -*-
%%
%% push.tex
%% Created Thu Nov 19 15:57:12 AKST 2015
%% Copyright (C) 2015 by Raymond E. Marcil <marcilr@gmail.com>
%%
%% Push
%%

%% ============================ Push =============================
%% ============================ Push =============================
\subsection{Push}

Update remote refs along with associated objects
\\
\\
\FIXME{Need for data here}
\\
\\
You can only push to two types of URL addresses:\footnote{About remote repositories,
\href{https://help.github.com/articles/about-remote-repositories/}{https://help.github.com/articles/about-remote-repositories/}}
\\
\\
$>$ An HTTPS URL like \cmd{https://github.com/user/repo.git}\\
$>$ An SSH URL, like \cmd{git@github.com:user/repo.git}\\






%% ========================== init ===============================
%% ========================== init ===============================
\clearpage
\subsubsection{init}
\cmd{\$ repo init -u <URL> [<OPTIONS>]}
\\
Installs Repo in the current directory. This creates a \cmd{.repo/}
directory that contains Git repositories for the Repo source code
and the standard Android manifest files.  The \cmd{.repo/}
directory also contains \cmd{manifest.xml}, which is a symlink to the
selected manifest in the \cmd{.repo/manifests/} directory.\footnote{Repo command reference --
\href{https://source.android.com/source/using-repo.html\#forall}{https://source.android.com/source/using-repo.html}}\\

\begin{table}[htb]
\begin{center}
\begin{tabular}{|p{.15\textwidth}|p{.75\textwidth}|}\hline
\centering Command&\centering Description\tabularnewline
\hline
\centering -u&Specify a URL from which to retrieve a manifest repository.
The common manifest can be found at:
\cmd{https://android.googlesource.com/platform/manifest}\\
\centering -m&Select a manifest file within the repository. If no manifest
 name is selected, the default is default.xml.\\
\centering -b&Specify a revision, i.e., a particular manifest-branch.\\
\hline
\end{tabular}
\caption {init Options}
\label{table:init_options}
\end{center}
\end{table}

%% ======================== Examples =============================
%% ======================== Examples =============================
\subsubsection{Examples}
This will create a new place to hold your local copy of the source
tree.   ``url'' should point to a Manifest repository that
describes the whole sources.  It is a special project with a file
(default.xml) that lists all the projects that Android is made of.
In the Manifest file, each projects has attributes about: where to
place it in the tree, where to download it from (git server),
revision that will be used (usually a branch name, tag or commit
sha-id).\footnote{Repo: Tips \& Tricks,\\
\href{http://xda-university.com/as-a-developer/repo-tips-tricks}{http://xda-university.com/as-a-developer/repo-tips-tricks}}
\\
\\
\indent\cmd{repo init -u <url> -b <branch>}
\\
\\
\noindent Note: For all remaining Repo commands, the current working
directory must either be the parent directory of \cmd{.repo/}
or a subdirectory of the parent directory.\footnote{Repo command reference --
\href{https://source.android.com/source/using-repo.html\#forall}{https://source.android.com/source/using-repo.html}}\\
\\
\\
\noindent \FIXME{Need example of GitHub checkout}
