%% -*- Mode: LaTeX -*-
%%
%% repo.tex
%% Created Tue Nov 17 15:18:14 AKST 2015
%% Copyright (C) 2015 by Raymond E. Marcil <marcilr@gmail.com>
%%
%% Google'd repo command
%%

%% ========================== Repo ===============================
%% ========================== Repo ===============================
%% ========================== Repo ===============================
\newpage
\section{Repo}
``Repo is a repository management tool that we built on top of Git.
Repo unifies the many Git repositories when necessary, does the
uploads to a revision control system, and automates parts of the
development workflow.  Repo is not meant to replace Git, only to
make it easier to work with Git in the context of Android.  The
\cmd{repo} command is an executable Python script that you can put
anywhere in your path.  In working with source files, you will use
Repo for across-network operations.  For example, with a single
Repo command you can download files from multiple repositories
into your local working directory.''\footnote{Developing -- \href{http://source.android.com/source/developing.html}{http://source.android.com/source/developing.html}}
\\
\\
\noindent \FIXME{The above \cmd{repo} quote has been heavily modified.
Need to rewrite with original verbage.}

%% -*- Mode: LaTeX -*-
%%
%% configuration.tex
%% Created Wed Nov 18 15:00:41 AKST 2015
%% Copyright (C) 2015 by Raymond E. Marcil <marcilr@gmail.com>
%%
%% Configuration
%%

%% ===================== Configuration ===========================
%% ===================== Configuration ===========================
\subsection{Configuration}
The \cmd{repo} utility requires a \cmd{.repo} directory containing
\cmd{manifest.xml} listing repositories to manage.

%% .repo directory
%% -*- Mode: LaTeX -*-
%%
%% repo.tex
%% Created Wed Nov 18 15:00:41 AKST 2015
%% Copyright (C) 2015 by Raymond E. Marcil <marcilr@gmail.com>
%%
%% .repo directory
%%

%% ==================== .repo directory ===========================
%% ==================== .repo directory ===========================
\subsubsection{.repo directory}
The \cmd{.repo/} directory, located in the repository base,
holds repo configuration.  The configuration includes a manifest
with information about all the projects and where their associated git
repositories are located.
\\
\\
Files within the \cmd{.repo/} subdirectory includes:
\begin{verbatim}
  manifests/
  manifests.git
  manifest.xml -> manifests/default.xml
  project-objects
  projects/
  repo/
\end{verbatim}
 
\noindent To create the \cmd{.repo/} subdirectory:

\begin{Verbatim}
 $ cd <my_repo>
 $ mkdir .repo/
 $ 
\end{Verbatim}

%% Manifest
%% -*- Mode: LaTeX -*-
%%
%% manifest.tex
%% Created Fri Nov 20 11:04:59 AKST 2015
%% Copyright (C) 2015 by Raymond E. Marcil <marcilr@gmail.com>
%%
%% Manifest
%%


%% ======================== Manifest =============================
%% ======================== Manifest =============================
\clearpage
\newpage
\subsubsection{Manifest}
The repo keeps a manifest, ``within the hidden directory named `.repo','' 
in ``a git project named `\cmd{manifests}' which
usually contains a file named `\cmd{default.xml}'.  This file
contains information about all the projects and where their associated
git repositories are located.  This file is also versioned thus when
you use the `\cmd{repo init -b XYZ}' command it will be reverted and you
can back to older branches that may have added/removed git projects
compared to the head.''\footnote{How does the Android repo manifest repository work?\\
\href{http://stackoverflow.com/questions/6149725/how-does-the-android-repo-manifest-repository-work}
{http://stackoverflow.com/questions/6149725/how-does-the-android-repo-manifest-repository-work}}
\\
\\
The \cmd{default.xml} file is symlinked to \cmd{.repo/manifest.xml} and
is created when the repo was initialized using:
\\
\\
\cmd{repo init -u $<$manifest path$>$}
\\
\\

%% ======================== Examples =============================
%% ======================== Examples =============================
\noindent \begin{bf}Examples\end{bf}
\\
\\
Following is a manifest, in \cmd{.repo/manifests/default.xml} file, showing use
of GitHub with username, ssh:// URL syntax, and 3 project repos
with different usernames:\footnote{Keiji Ariyama, \href{https://github.com/keiji/repo-sample/blob/master/default.xml}{https://github.com/keiji/repo-sample/blob/master/default.xml}}

\begin{Verbatim}
 <?xml version="1.0" encoding="UTF-8"?>
 <manifest>
   <remote name="origin" fetch="ssh://git@github.com/" />
   <default revision="master" remote="origin" />

   <project path="lib/plist-lib"
            name="keiji/AndroidPListLib.git" remote="origin" />

   <project path="lib/json-pull-parser"
            name="vvakame/JsonPullParser.git" remote="origin" />

   <project path="apps/twicca_megane_plugin"
            name="zaki50/TwiccaMeganePlugin.git" remote="origin" />
 </manifest>
\end{Verbatim}



%% ================== .repo/ subdirectory ===========================
%% ================== .repo/ subdirectory ===========================
\subsection{\cmd{.repo/} subdirectory}
The \cmd{.repo/} subdirectory, located in the repository base,
holds repo configuration.  The configuration includes a manifest
with information about all the projects and where their associated git
repositories are located.
\\
\\
Files within the \cmd{.repo/} subdirectory includes:
\begin{verbatim}
  manifests/
  manifests.git
  manifest.xml -> manifests/default.xml
  project-objects
  projects/
  repo/
\end{verbatim}
 
\noindent To create the \cmd{.repo/} subdirectory:

\begin{Verbatim}
 $ cd <my_repo>
 $ mkdir .repo/
 $ 
\end{Verbatim}

%% ======================== Manifest =============================
%% ======================== Manifest =============================
\subsection{Manifest}
The repo keeps a manifest, ``within the hidden directory named `.repo','' 
in ``a git project named `\cmd{manifests}' which
usually contains a file named `\cmd{default.xml}'.  This file
contains information about all the projects and where their associated
git repositories are located.  This file is also versioned thus when
you use the `\cmd{repo init -b XYZ}' command it will be reverted and you
can back to older branches that may have added/removed git projects
compared to the head.''\footnote{How does the Android repo manifest repository work?\\
\href{http://stackoverflow.com/questions/6149725/how-does-the-android-repo-manifest-repository-work}
{http://stackoverflow.com/questions/6149725/how-does-the-android-repo-manifest-repository-work}}
\\
\\
The \cmd{default.xml} file is symlinked to \cmd{.repo/manifest.xml} and
is created when the repo was initialized using:
\\
\\
\cmd{repo init -u $<$manifest path$>$}


%% ======================== Examples =============================
%% ======================== Examples =============================
\subsubsection{Examples}
Following is a manifest, in \cmd{.repo/manifests/default.xml} file, showing use
of GitHub with username, ssh:// URL syntax, and 3 project repos
with different usernames:\footnote{Keiji Ariyama, \href{https://github.com/keiji/repo-sample/blob/master/default.xml}{https://github.com/keiji/repo-sample/blob/master/default.xml}}

\begin{Verbatim}
 <?xml version="1.0" encoding="UTF-8"?>
 <manifest>
   <remote name="origin" fetch="ssh://git@github.com/" />
   <default revision="master" remote="origin" />

   <project path="lib/plist-lib"
            name="keiji/AndroidPListLib.git" remote="origin" />

   <project path="lib/json-pull-parser"
            name="vvakame/JsonPullParser.git" remote="origin" />

   <project path="apps/twicca_megane_plugin"
            name="zaki50/TwiccaMeganePlugin.git" remote="origin" />
 </manifest>
\end{Verbatim}


%% ======================== Commands =============================
%% ======================== Commands =============================
\newpage
\subsection{Commands}
Repo usage takes the following form:\footnote{Repo command reference\\
\href{https://source.android.com/source/using-repo.html\#help}{https://source.android.com/source/using-repo.html\#help}}
\\
\indent \cmd{repo <COMMAND> <OPTIONS>}
\\
\\
Optional elements are shown in brackets [ ].  For example, many commands take
a project list as an argument.  You can specify project-list as a list of
names or a list of paths to local source directories for the projects:

\indent \cmd{repo sync [<PROJECT0> <PROJECT1> <PROJECTN>]}\\
\indent \cmd{repo sync [</PATH/TO/PROJECT0> ... </PATH/TO/PROJECTN>]}
\\

\noindent Once Repo is installed, you can find the latest documentation
starting with a summary of all commands by running:
\\
\indent \cmd{repo help}
\\
\\
\noindent You can get information about any command by running this within a Repo tree:
\\
\indent \cmd{repo help <COMMAND>}
\\
\\
NOTE: For \cmd{repo} commands without syntax here see the 
Repo command reference.\footnote{Ibid.}


\begin{table}[htb]
\begin{center}
\begin{tabular}{|p{.25\textwidth}|p{.47\textwidth}|}\hline
Command&Description\\
\hline
abandon&Permanently abandon a development branch\\
branch&View current topic branches\\
branches&View current topic branches\\
checkout&Checkout a branch for development\\
cherry-pick&Cherry-pick a change\\
diff&Show changes between commit and working tree\\
diffmanifests&Manifest diff utility\\
download&Download and checkout a change\\
grep&Print lines matching a pattern\\
forall&Executes the given shell command in each project.\footnote{Repo command reference --
\href{https://source.android.com/source/using-repo.html\#forall}{https://source.android.com/source/using-repo.html\#forall}}\\
help&Display detailed help on a command\\
info&Get info on the manifest branch, current branch or unmerged branches\\
init&Install repo in the current working directory\\
list&List projects and their associated directories\\
overview&Display overview of unmerged project branches\\
prune&Prune (delete) already merged topics\\
rebase&Rebase local branches on upstream branch\\
start& Start a new branch for development\\
status&Show the working tree status\\
sync&Update working tree to the latest revision\\
upload&Upload changes for code review\\
\hline
\end{tabular}
\caption {Repo Commands}
\label{table:commands}
\end{center}
\end{table}



%% ========================== init ===============================
%% ========================== init ===============================
\clearpage
\subsubsection{init}
\cmd{\$ repo init -u <URL> [<OPTIONS>]}
\\
Installs Repo in the current directory. This creates a \cmd{.repo/}
directory that contains Git repositories for the Repo source code
and the standard Android manifest files.  The \cmd{.repo/}
directory also contains \cmd{manifest.xml}, which is a symlink to the
selected manifest in the \cmd{.repo/manifests/} directory.\footnote{Repo command reference --
\href{https://source.android.com/source/using-repo.html\#forall}{https://source.android.com/source/using-repo.html}}\\

\begin{table}[htb]
\begin{center}
\begin{tabular}{|p{.15\textwidth}|p{.75\textwidth}|}\hline
\centering Command&\centering Description\tabularnewline
\hline
\centering -u&Specify a URL from which to retrieve a manifest repository.
The common manifest can be found at:
\cmd{https://android.googlesource.com/platform/manifest}\\
\centering -m&Select a manifest file within the repository. If no manifest
 name is selected, the default is default.xml.\\
\centering -b&Specify a revision, i.e., a particular manifest-branch.\\
\hline
\end{tabular}
\caption {init Options}
\label{table:init_options}
\end{center}
\end{table}

%% ======================== Examples =============================
%% ======================== Examples =============================
\subsubsection{Examples}
This will create a new place to hold your local copy of the source
tree.   ``url'' should point to a Manifest repository that
describes the whole sources.  It is a special project with a file
(default.xml) that lists all the projects that Android is made of.
In the Manifest file, each projects has attributes about: where to
place it in the tree, where to download it from (git server),
revision that will be used (usually a branch name, tag or commit
sha-id).\footnote{Repo: Tips \& Tricks,\\
\href{http://xda-university.com/as-a-developer/repo-tips-tricks}{http://xda-university.com/as-a-developer/repo-tips-tricks}}
\\
\\
\indent\cmd{repo init -u <url> -b <branch>}
\\
\\
\noindent Note: For all remaining Repo commands, the current working
directory must either be the parent directory of \cmd{.repo/}
or a subdirectory of the parent directory.\footnote{Repo command reference --
\href{https://source.android.com/source/using-repo.html\#forall}{https://source.android.com/source/using-repo.html}}\\
\\
\\
\noindent \FIXME{Need example of GitHub checkout}
