%% -*- Mode: LaTeX -*-
%%
%% remotes.tex
%% Created Wed Nov 18 15:00:41 AKST 2015
%% Copyright (C) 2015 by Raymond E. Marcil <marcilr@gmail.com>
%%
%% Remotes
%%

%% ========================= Remotes =============================
%% ========================= Remotes =============================
\newpage
\subsection{Remotes}
``To be able to collaborate on any Git project, you need to know
how to manage your remote repositories.  Remote repositories are
versions of your project that are hosted on the Internet or
network somewhere.  You can have several of them, each of which
generally is either read-only or read/write for you.
Collaborating with others involves managing these remote
repositories and pushing and pulling data to and from them when
you need to share work.  Managing remote repositories includes
knowing how to add remote repositories, remove remotes that are no
longer valid, manage various remote branches and define them as
being tracked or not, and more. In this section, we'll cover some
of these remote-management skills.''\footnote{Git Basics - Working
with Remotes,\\
\href{http://git-scm.com/book/en/v2/Git-Basics-Working-with-Remotes}{http://git-scm.com/book/en/v2/Git-Basics-Working-with-Remotes}}

%% Remote Repositories
%% -*- Mode: LaTeX -*-
%%
%% repositories.tex
%% Created Thu Nov 19 11:00:57 AKST 2015
%% Copyright (C) 2015 by Raymond E. Marcil <marcilr@gmail.com>
%%
%% Introduction
%%

%% ====================== Repositories ===========================
%% ====================== Repositories ===========================
%% ====================== Repositories ===========================
\newpage
\section{Repositories}


%% Remote Commands
%% -*- Mode: LaTeX -*-
%%
%% commands.tex
%% Created Wed Nov 18 15:00:41 AKST 2015
%% Copyright (C) 2015 by Raymond E. Marcil <marcilr@gmail.com>
%%
%% Commands
%%

%% ===================== Commands =======================
%% ===================== Commands =======================
%% ===================== Commands =======================
\newpage
\subsection{Commands}

\begin{table}[htb]
\begin{center}
\begin{tabular}{|p{.15\textwidth}|p{.70\textwidth}|}\hline
Command&Description\\
\hline
\cmd{add}&Add file contents to the index\\
\cmd{apply}&Apply a patch to files and/or to the index\\
\cmd{clone}&Get a complete copy of a repository\\
\cmd{commit}&Record changes to the repository\\
\cmd{pull}&Fetch from and integrate with another repository or a local branch\\
\cmd{push}&Update remote refs along with associated objects\\
\cmd{rebase}&Forward-port local commits to the updated upstream head\\
\cmd{status}&Show the working tree status\\
\hline
\end{tabular}
\caption {Commands}
\label{table:commands}
\end{center}
\end{table}


%% Add
%% -*- Mode: LaTeX -*-
%%
%% add.tex
%% Created Wed Nov 18 15:00:41 AKST 2015
%% Copyright (C) 2015 by Raymond E. Marcil <marcilr@gmail.com>
%%
%% Add
%%

%% ============================ Add ==============================
%% ============================ Add ==============================
\subsubsection{Add}
Add file contents to the index
\\
\\
This command updates the index using the current content found in
the working tree, to prepare the content staged for the next
commit.  It typically adds the current content of existing paths
as a whole, but with some options it can also be used to add
content with only part of the changes made to the working tree
files applied, or remove paths that do not exist in the working
tree anymore.\footnote{git-add - Add file contents to the index\\
\href{https://git-scm.com/docs/git-add}{https://git-scm.com/docs/git-add}}




%% ========================= Examples ============================
%% ========================= Examples ============================
\subsubsection{Examples}
Create and add \cmd{.repo/manifests/default.xml} file and
\cmd{.repo/manifests/manifest.xml}
\begin{Verbatim}
 covellite:~/git/.repo/manifests$ git add default.xml
 covellite:~/git/.repo/manifests$ git commit default.xml
 covellite:~/git/.repo/manifests$ cd ..
 covellite:~/git/.repo/$ ln -s manifests/default.xml manifest.xml
 covellite:~/git/.repo/$ git add manifest.xml
 covellite:~/git/.repo/$ git commit
\end{Verbatim}


%% Clone
%% -*- Mode: LaTeX -*-
%%
%% clone.tex
%% Created Wed Nov 18 15:00:41 AKST 2015
%% Copyright (C) 2015 by Raymond E. Marcil <marcilr@gmail.com>
%%
%% Clone
%%

%% =========================== Clone =============================
%% =========================== Clone =============================
\newpage
\subsubsection{Clone}
To grab a complete copy of another user's repository, use git clone like this:

\begin{Verbatim}
  $ git clone https://github.com/USERNAME/REPOSITORY.git
  # Clones a repository to your computer
\end{Verbatim}

\noindent When you run \cmd{git clone}, the following actions occur:\\
$>$ A new folder called repo is made\\
$>$ It is initialized as a Git repository\\
$>$ A remote named origin is created, pointing to the URL you cloned from\\
$>$ All of the repository's files and commits are downloaded there\\
$>$ The default branch (usually called master) is checked out\\

\noindent For every branch \cmd{foo} in the remote repository, a corresponding remote-tracking
branch \cmd{refs/remotes/origin/foo} is created in your local repository.  You can
usually abbreviate such remote-tracking branch names to \cmd{origin/foo}.\footnote{Fetching a remote,
\cmd{git clone}, \cmd{git fetch}, \cmd{git merge}, \cmd{git pull},
\href{https://help.github.com/articles/fetching-a-remote/}{https://help.github.com/articles/fetching-a-remote/}}


%% ========================= Examples ============================
%% ========================= Examples ============================
\subsubsection{Examples}
To clone repository named \cmd{git} from GitHub to local \cmd{covellite} workstation:
\begin{Verbatim}
 covellite:~$ git clone https://github.com/marcilr/git.git
 Cloning into 'git'...
 warning: You appear to have cloned an empty repository.
 Checking connectivity... done.
 covellite:~$
\end{Verbatim}

\noindent Clone \cmd{.repo} repository into \cmd{git/} directory:
\begin{Verbatim}
 covellite:~/git$ git clone https://github.com/marcilr/.repo
 Cloning into '.repo'...
 warning: You appear to have cloned an empty repository.
 Checking connectivity... done.
 covellite:~/git$
\end{Verbatim}


\newpage
\noindent To clone a Git repository over SSH, you can specify ssh:// URL like this:\\
\begin{Verbatim}
 $ git clone ssh://user@server/project.git
\end{Verbatim}

\noindent Or you can use the shorter scp-like syntax for the SSH protocol:

\begin{Verbatim}
 $ git clone user@server:project.git
\end{Verbatim}

\noindent You can also not specify a user, and Git assumes the user
you're currently logged in as.\footnote{
Git on the Server - The Protocols, The SSH Protocol,\\
\href{https://git-scm.com/book/en/v2/Git-on-the-Server-The-Protocols}{https://git-scm.com/book/en/v2/Git-on-the-Server-The-Protocols}}
\\
\\
\noindent \FIXME{Need more commands here.}


%% Push
%% -*- Mode: LaTeX -*-
%%
%% push.tex
%% Created Thu Nov 19 15:57:12 AKST 2015
%% Copyright (C) 2015 by Raymond E. Marcil <marcilr@gmail.com>
%%
%% Push
%%

%% ============================ Push =============================
%% ============================ Push =============================
\subsection{Push}

Update remote refs along with associated objects
\\
\\
\FIXME{Need for data here}
\\
\\
You can only push to two types of URL addresses:\footnote{About remote repositories,
\href{https://help.github.com/articles/about-remote-repositories/}{https://help.github.com/articles/about-remote-repositories/}}
\\
\\
$>$ An HTTPS URL like \cmd{https://github.com/user/repo.git}\\
$>$ An SSH URL, like \cmd{git@github.com:user/repo.git}\\


