%% -*- Mode: LaTeX -*-
%%
%% remotes.tex
%% Created Wed Nov 18 15:00:41 AKST 2015
%% Copyright (C) 2015 by Raymond E. Marcil <marcilr@gmail.com>
%%
%% Remotes
%%

%% ========================= Remotes =============================
%% ========================= Remotes =============================
\newpage
\subsection{Remote Repositories}
``To be able to collaborate on any Git project, you need to know
how to manage your remote repositories.  Remote repositories are
versions of your project that are hosted on the Internet or
network somewhere.  You can have several of them, each of which
generally is either read-only or read/write for you.
Collaborating with others involves managing these remote
repositories and pushing and pulling data to and from them when
you need to share work.  Managing remote repositories includes
knowing how to add remote repositories, remove remotes that are no
longer valid, manage various remote branches and define them as
being tracked or not, and more. In this section, we'll cover some
of these remote-management skills.''\footnote{Git Basics - Working
with Remotes,\\
\href{http://git-scm.com/book/en/v2/Git-Basics-Working-with-Remotes}{http://git-scm.com/book/en/v2/Git-Basics-Working-with-Remotes}}
\\
\\
\noindent GitHub's collaborative approach to development depends on
publishing commits from your local repository for other people to
view, fetch, and update.\footnote{About remote repositories,
\href{https://help.github.com/articles/about-remote-repositories/}{https://help.github.com/articles/about-remote-repositories/}}
\\
\\
A remote URL is Git's fancy way of saying ``the place where your
code is stored.'' That URL could be your repository on GitHub, or
another user's fork, or even on a completely different server.
\\
\\
You can only push to two types of URL addresses:
\\
\\
$>$ An HTTPS URL like \cmd{https://github.com/user/repo.git}\\
$>$ An SSH URL, like \cmd{git@github.com:user/repo.git}\\
\\
Git associates a remote URL with a name, and your default remote is usually called origin.


%% Remote Commands
%% -*- Mode: LaTeX -*-
%%
%% commands.tex
%% Created Wed Nov 18 15:00:41 AKST 2015
%% Copyright (C) 2015 by Raymond E. Marcil <marcilr@gmail.com>
%%
%% Commands
%%

%% ===================== Commands =======================
%% ===================== Commands =======================
%% ===================== Commands =======================
\newpage
\section{Commands}

\FIXME{Need to update with commands (log, status, mv, rm, status, and commit from:\\
\href{http://getliner.com/qXYnW/}{http://getliner.com/qXYnW/}}

\begin{table}[htb]
\begin{center}
\begin{tabular}{|p{.15\textwidth}|p{.70\textwidth}|}\hline
Command&Description\\
\hline
\cmd{add}&Add file contents to the index\\
\cmd{apply}&Apply a patch to files and/or to the index\\
\cmd{clone}&Get a complete copy of a repository\\
\cmd{commit}&Record changes to the repository\\
\cmd{diff}&Show changes between commits, commit and working tree, etc\\
\cmd{init}&Initialize new repository\\
\cmd{pull}&Fetch from and integrate with another repository or a local branch\\
\cmd{push}&Update remote refs along with associated objects\\
\cmd{rebase}&Forward-port local commits to the updated upstream head\\
\cmd{status}&Show the working tree status.  See what files have changed.\\
\hline
\end{tabular}
\caption {Commands}
\label{table:commands}
\end{center}
\end{table}


%% Sections
%% ========
%% Add
%% Clone
%% Commit
%% Diff
%% Init
%% Push
%% git remote add
%% git remote set-url
%% git remote rename
%% git remote rm
%%
%% -*- Mode: LaTeX -*-
%%
%% add.tex
%% Created Wed Nov 18 15:00:41 AKST 2015
%% Copyright (C) 2015 by Raymond E. Marcil <marcilr@gmail.com>
%%
%% Add
%%

%% ============================ Add ==============================
%% ============================ Add ==============================
\subsubsection{Add}
Add file contents to the index
\\
\\
This command updates the index using the current content found in
the working tree, to prepare the content staged for the next
commit.  It typically adds the current content of existing paths
as a whole, but with some options it can also be used to add
content with only part of the changes made to the working tree
files applied, or remove paths that do not exist in the working
tree anymore.\footnote{git-add - Add file contents to the index\\
\href{https://git-scm.com/docs/git-add}{https://git-scm.com/docs/git-add}}
\\
\\

%% ========================= Examples ============================
%% ========================= Examples ============================
\noindent \begin{bf}Examples\end{bf}
\\
\\
\noindent Create and add \cmd{.repo/manifests/default.xml} file and
\cmd{.repo/manifests/manifest.xml}
\begin{Verbatim}
 covellite:~/git/.repo/manifests$ git add default.xml
 covellite:~/git/.repo/manifests$ git commit default.xml
 covellite:~/git/.repo/manifests$ cd ..
 covellite:~/git/.repo/$ ln -s manifests/default.xml manifest.xml
 covellite:~/git/.repo/$ git add manifest.xml
 covellite:~/git/.repo/$ git commit
\end{Verbatim}

%% -*- Mode: LaTeX -*-
%%
%% clone.tex
%% Created Wed Nov 18 15:00:41 AKST 2015
%% Copyright (C) 2015 by Raymond E. Marcil <marcilr@gmail.com>
%%
%% Clone
%%

%% =========================== Clone =============================
%% =========================== Clone =============================
\newpage
\subsection{Clone}
To grab a complete copy of another user's repository, use git clone like this:

\begin{Verbatim}
  $ git clone https://github.com/USERNAME/REPOSITORY.git
  # Clones a repository to your computer
\end{Verbatim}

\noindent When you run \cmd{git clone}, the following actions occur:\\
$>$ A new folder called repo is made\\
$>$ It is initialized as a Git repository\\
$>$ A remote named origin is created, pointing to the URL you cloned from\\
$>$ All of the repository's files and commits are downloaded there\\
$>$ The default branch (usually called master) is checked out\\

\noindent For every branch \cmd{foo} in the remote repository, a corresponding remote-tracking
branch \cmd{refs/remotes/origin/foo} is created in your local repository.  You can
usually abbreviate such remote-tracking branch names to \cmd{origin/foo}.\footnote{Fetching a remote,
\cmd{git clone}, \cmd{git fetch}, \cmd{git merge}, \cmd{git pull},
\href{https://help.github.com/articles/fetching-a-remote/}{https://help.github.com/articles/fetching-a-remote/}}

%% ========================== Syntax =============================
%% ========================== Syntax =============================
\vspace{20pt}
\noindent\begin{bf}Syntax\end{bf}
\\
\\
\noindent Clone the required package by executing the following
command:\footnote{Cloning Tizen Source,\\
\href{https://source.tizen.org/documentation/developer-guide/getting-started-guide/cloning-tizen-source}{https://source.tizen.org/documentation/developer-guide/getting-started-guide/cloning-tizen-source}}

\begin{Verbatim}
 $ git clone [-b <Branch>] ssh://<Username>@review.tizen.org:\
 29418/<Gerrit_Project> [<Local_Project>]
\end{Verbatim}

or

\begin{Verbatim}
 $ git clone ssh://<Username>@review.tizen.org:29418/pkgs/a/avsystem
\end{Verbatim}

%% ========================= Examples ============================
%% ========================= Examples ============================
\vspace{20pt}
\noindent\begin{bf}Examples\end{bf}
\\
\\
To clone repository named \cmd{git} from GitHub to local \cmd{covellite} workstation:
\begin{Verbatim}
 covellite:~$ git clone https://github.com/marcilr/git.git
 Cloning into 'git'...
 warning: You appear to have cloned an empty repository.
 Checking connectivity... done.
 covellite:~$
\end{Verbatim}

\noindent Clone \cmd{.repo} repository into \cmd{git/} directory:
\begin{Verbatim}
 covellite:~/git$ git clone https://github.com/marcilr/.repo
 Cloning into '.repo'...
 warning: You appear to have cloned an empty repository.
 Checking connectivity... done.
 covellite:~/git$
\end{Verbatim}

\noindent To clone a Git repository over SSH, you can specify ssh:// URL like this:\\
\begin{Verbatim}
 $ git clone ssh://user@server/project.git
\end{Verbatim}

\noindent Or you can use the shorter scp-like syntax for the SSH protocol:

\begin{Verbatim}
 $ git clone user@server:project.git
\end{Verbatim}

\noindent You can also not specify a user, and Git assumes the user
you're currently logged in as.\footnote{
Git on the Server - The Protocols, The SSH Protocol,\\
\href{https://git-scm.com/book/en/v2/Git-on-the-Server-The-Protocols}{https://git-scm.com/book/en/v2/Git-on-the-Server-The-Protocols}}
\\
\\
\noindent \FIXME{Need more commands here.}

%% -*- Mode: LaTeX -*-
%%
%% commit.tex
%% Created Wed Dec 23 11:20:19 AKST 2015
%% Copyright (C) 2015 by Raymond E. Marcil <marcilr@gmail.com>
%%
%% Commit - Record changes to the repository
%%

%% ============================ Commit ==============================
%% ============================ Commit ==============================
\newpage
\subsection{Commit}
Record changes to the repository
\\
\\
Stores the current contents of the index in a new commit along with
a log message from the user describing the changes.\footnote{git-commit - Record changes to the repository,\\
\href{https://git-scm.com/docs/git-commit}{https://git-scm.com/docs/git-commit}}
\\
\\

%% ========================= Examples ============================
%% ========================= Examples ============================
\noindent \begin{bf}Examples\end{bf}
\\
\\
\noindent Commit \cmd{spectracom.txt} file with message
``\cmd{Added GCI Network Services OSS comments.}''
\begin{Verbatim}
 $ git commit -m "Added GCI Network Services OSS comments." spectracom.txt
 [master 5267495] Added GCI Network Services OSS comments.
  1 file changed, 13 insertions(+)
 $ 
\end{Verbatim}

%% -*- Mode: LaTeX -*-
%%
%% diff.tex
%% Created Fri Nov 20 09:27:47 AKST 2015
%% Copyright (C) 2015 by Raymond E. Marcil <marcilr@gmail.com>
%%
%% Diff
%%

%% ============================ Diff ==============================
%% ============================ Diff ==============================
\subsection{Diff}
Show changes between commits, commit and working tree, etc.
\\
\\
\cmd{git diff [file]} to see exactly what I modified.
\\
\\
Show changes between the working tree and the index or a tree,
changes between the index and a tree, changes between two trees,
changes between two blob objects, or changes between two files on
disk.\footnote{git-diff - Show changes between commits, commit and working tree, etc\\
\href{https://git-scm.com/docs/git-diff}{https://git-scm.com/docs/git-diff}}
\\
\\
\begin{bf}Synopsis\end{bf}
\begin{verbatim}
 git diff [options] [<commit>] [--] [<path>...]
 git diff [options] --cached [<commit>] [--] [<path>...]
 git diff [options] <commit> <commit> [--] [<path>...]
 git diff [options] <blob> <blob>
 git diff [options] [--no-index] [--] <path> <path>
\end{verbatim}

%% -*- Mode: LaTeX -*-
%%
%% init.tex
%% Created Wed Dec  2 13:50:35 AKST 2015
%% Copyright (C) 2015 by Raymond E. Marcil <marcilr@gmail.com>
%%
%% Init
%%

%% ============================ Init ==============================
%% ============================ Init ==============================
\subsection{Init}
Initialize new repository
\\
\\
The \cmd{git init} command creates a new Git repository.  It can be
used to convert an existing, unversioned project to a Git
repository or initialize a new empty repository. Most of the other
Git commands are not available outside of an initialized
repository, so this is usually the first command you’ll run in a
new project.
\\
\\
Executing \cmd{git init} creates a \cmd{.git} subdirectory in the
project root, which contains all of the necessary metadata for the
repo. Aside from the .git directory, an existing project remains
unaltered (unlike SVN, Git doesn't require a \cmd{.git} folder in
every subdirectory).\footnote{Setting up a repository, git init,\\
\href{https://www.atlassian.com/git/tutorials/setting-up-a-repository}
{https://www.atlassian.com/git/tutorials/setting-up-a-repository}}
\\
\\
\begin{bf}Examples\end{bf}

\begin{Verbatim}
 $ mkdir wti
 $ cd wti/
 $ git init
 Initialized empty Git repository in /home/marcilr/wti/.git/
 $ 
\end{Verbatim}


%% -*- Mode: LaTeX -*-
%%
%% push.tex
%% Created Thu Nov 19 15:57:12 AKST 2015
%% Copyright (C) 2015 by Raymond E. Marcil <marcilr@gmail.com>
%%
%% Push
%%

%% ============================ Push =============================
%% ============================ Push =============================
\subsubsection{Push}

Update remote refs along with associated objects
\\
\\
\FIXME{Need for data here}
\\
\\
You can only push to two types of URL addresses:\footnote{About remote repositories,
\href{https://help.github.com/articles/about-remote-repositories/}{https://help.github.com/articles/about-remote-repositories/}}
\\
\\
$>$ An HTTPS URL like \cmd{https://github.com/user/repo.git}\\
$>$ An SSH URL, like \cmd{git@github.com:user/repo.git}\\


%% -*- Mode: LaTeX -*-
%%
%% remote-add.tex
%% Created Thu Nov 19 08:25:10 AKST 2015
%% Copyright (C) 2015 by Raymond E. Marcil <marcilr@gmail.com>
%%
%% git remote add
%%


%% ===================== git remote add =========================
%% ===================== git remote add =========================
\newpage
\subsection{git remote add}
To add a new remote, use the \cmd{git remote add} command on the
terminal, in the directory your repository is stored at.\footnote{Adding a remote,
\href{https://help.github.com/articles/adding-a-remote/}{https://help.github.com/articles/adding-a-remote/}}
\\
\\
The \cmd{git remote add} command takes two arguments:
\vspace{-10pt}
\begin{verbatim}
$ git remote add <NAME> <REMOTE_URL> 
\end{verbatim}

\noindent Where:
\vspace{-10pt}
\begin{table}[htb]
%%\begin{table}
\begin{center}
\begin{tabular}{p{.20\textwidth}p{.80\textwidth}}
\hspace{20pt}\cmd{<NAME>}&- A remote name, for example, \cmd{origin}\\
\hspace{20pt}\cmd{<REMOTE\_URL>}&- A remote URL, for example, \cmd{https://github.com/user/repo.git}\\
\end{tabular}
\label{table:remote_add}
\end{center}
\end{table}

\vspace{-20pt}
\noindent Git associates a remote URL with a name, and your
default remote is usually called origin.\footnote{About remote repositories,
\href{https://help.github.com/articles/about-remote-repositories/}{https://help.github.com/articles/about-remote-repositories/}}
\\

%% ======================== Examples ============================
%% ======================== Examples ============================
\vspace{10pt}
\noindent \begin{bf}Examples\end{bf}
\begin{Verbatim}
 # This associates the name origin with SSH URL for repo.git repository.
 $ git remote add origin git@github.com:user/repo.git
\end{Verbatim}

\noindent Alternatively using https syntax:

\begin{Verbatim}
 $ git remote add origin https://github.com/user/repo.git
 # Set a new remote

 $ git remote -v
 # Verify new remote
 origin  https://github.com/user/repo.git (fetch)
 origin  https://github.com/user/repo.git (push)
\end{Verbatim}

%% -*- Mode: LaTeX -*-
%%
%% get-remote-set-url.tex
%% Created Thu Nov 19 08:25:10 AKST 2015
%% Copyright (C) 2015 by Raymond E. Marcil <marcilr@gmail.com>
%%
%% git remote set url
%%


%% =================== git remote set-url =======================
%% =================== git remote set-url =======================
\newpage
\subsubsection{git remote set-url}
The \cmd{git remote set-url} command changes an existing remote
repository URL.\footnote{Changing a remote's URL,
\href{https://help.github.com/articles/changing-a-remote-s-url/}{https://help.github.com/articles/changing-a-remote-s-url/}}
\\
\\
The git remote set-url command takes two arguments:
\\
$>$ An existing remote name. For example, \cmd{origin} or \cmd{upstream} are two common choices.\\
$>$ A new URL for the remote. For example:\\

\noindent\hspace*{10pt}$>$ If you're updating to use HTTPS, your URL might look like:\\
\noindent\hspace*{23pt}\cmd{https://github.com/USERNAME/OTHERREPOSITORY.git}\\
\\
\noindent\hspace*{10pt}$>$ If you're updating to use SSH, your URL might look like:\\
\noindent\hspace*{23pt}\cmd{git@github.com:USERNAME/OTHERREPOSITORY.git}


%% ========================== Examples =========================
%% ========================== Examples =========================
\vspace{20pt}
\noindent\begin{bf}Examples\end{bf}
\\

%% ========= Switching remote URLs from SSH to HTTPS ===========
%% ========= Switching remote URLs from SSH to HTTPS ===========
\noindent\begin{bf}Switching remote URLs from SSH to HTTPS\end{bf}
\begin{enumerate}
  \item{Open Terminal (for Mac and Linux users) or the command prompt
       (for Windows users).\footnote{Switching remote URLs from HTTPS to SSH,
       \href{https://help.github.com/articles/changing-a-remote-s-url/}{https://help.github.com/articles/changing-a-remote-s-url/}}}
  \item{Change the current working directory to your local project.}
  \item{List your existing remotes in order to get the name of the remote you want to change.
\begin{Verbatim}
 $ git remote -v
 # origin  git@github.com:USERNAME/REPOSITORY.git (fetch)
 # origin  git@github.com:USERNAME/REPOSITORY.git (push)
\end{Verbatim}
       }
  \item{Change your remote's URL from SSH to HTTPS with the git remote set-url command.
\begin{Verbatim}
 $ git remote set-url origin \
 https://github.com/USERNAME/OTHERREPOSITORY.git
\end{Verbatim}
      }

  \item{Verify that the remote URL has changed.
\begin{Verbatim}
$ git remote -v
# Verify new remote URL
# origin  https://github.com/USERNAME/OTHERREPOSITORY.git (fetch)
# origin  https://github.com/USERNAME/OTHERREPOSITORY.git (push)
\end{Verbatim}
       }
\end{enumerate}


%% ========= Switching remote URLs from HTTPS to SSH ===========
%% ========= Switching remote URLs from HTTPS to SSH ===========
\newpage
\begin{bf}Switching remote URLs from HTTPS to SSH\end{bf}
\begin{enumerate}
  \item{Open Terminal (for Mac and Linux users) or the command prompt
       (for Windows users).\footnote{Switching remote URLs from HTTPS to SSH,
       \href{https://help.github.com/articles/changing-a-remote-s-url/}{https://help.github.com/articles/changing-a-remote-s-url/}}}
  \item{Change the current working directory to your local project.}
  \item{List your existing remotes in order to get the name of the remote you want to change.
\begin{Verbatim}
 $ git remote -v
 origin  https://github.com/USERNAME/REPOSITORY.git (fetch)
 origin  https://github.com/USERNAME/REPOSITORY.git (push)
\end{Verbatim}
       }
  \item{Change your remote's URL from HTTPS to SSH with the git remote set-url command.
\begin{Verbatim}
 $ git remote set-url origin \
 git@github.com:USERNAME/OTHERREPOSITORY.git
\end{Verbatim}
      }

  \item{Verify that the remote URL has changed.
\begin{Verbatim}
 $ git remote -v
 # Verify new remote URL
 origin  git@github.com:USERNAME/OTHERREPOSITORY.git (fetch)
 origin  git@github.com:USERNAME/OTHERREPOSITORY.git (push)
\end{Verbatim}
       }
\end{enumerate}

%% -*- Mode: LaTeX -*-
%%
%% remote-rename.tex
%% Created Thu Nov 19 08:25:10 AKST 2015
%% Copyright (C) 2015 by Raymond E. Marcil <marcilr@gmail.com>
%%
%% git remote rename
%%


%% ==================== git remote rename ======================
%% ==================== git remote rename ======================
\newpage
\subsection{Remote rename}
Use the git remote rename command to rename an existing remote.\footnote{Renaming a remote,
\href{https://help.github.com/articles/renaming-a-remote/}{https://help.github.com/articles/renaming-a-remote/}}
\\
\\
The git remote rename command takes two arguments:
\\
\\
$>$ An existing remote name, for example, origin A new name for the\\
$>$ remote, for example, destination

\noindent \begin{bf}Example\end{bf}\\
The examples below assume you're cloning using HTTPS, which is recommended.

\begin{Verbatim}
 $ git remote -v
 # View existing remotes
 origin  https://github.com/OWNER/REPOSITORY.git (fetch)
 origin  https://github.com/OWNER/REPOSITORY.git (push)

 $ git remote rename origin destination
 # Change remote name from 'origin' to 'destination'

 $ git remote -v
 # Verify remote's new name
 destination  https://github.com/OWNER/REPOSITORY.git (fetch)
 destination  https://github.com/OWNER/REPOSITORY.git (push)
\end{Verbatim}

%% -*- Mode: LaTeX -*-
%%
%% remote-rm.tex
%% Created Thu Nov 19 08:25:10 AKST 2015
%% Copyright (C) 2015 by Raymond E. Marcil <marcilr@gmail.com>
%%
%% git remote rm
%%


%% ====================== git remote rm ======================
%% ====================== git remote rm ======================
\newpage
\subsection{git remote rm}
Use the \cmd{git remote rm} command to remove a remote URL
from your repository.\footnote{Removing a remote,
\href{https://help.github.com/articles/removing-a-remote/}{https://help.github.com/articles/removing-a-remote/}}
\\
\\
The \cmd{git remote rm} command takes one argument:
\\
\\
$>$ A remote name, for example, destination
\\
\\
\noindent \begin{bf}Example\end{bf}
\\
\\
The examples below assume you're cloning using HTTPS, which is recommended.

\begin{Verbatim}
 $ git remote -v
 # View current remotes
 origin  https://github.com/OWNER/REPOSITORY.git (fetch)
 origin  https://github.com/OWNER/REPOSITORY.git (push)
 destination  https://github.com/FORKER/REPOSITORY.git (fetch)
 destination  https://github.com/FORKER/REPOSITORY.git (push)

 $ git remote rm destination
 # Remove remote
 $ git remote -v
 # Verify it's gone
 origin  https://github.com/OWNER/REPOSITORY.git (fetch)
 origin  https://github.com/OWNER/REPOSITORY.git (push)
\end{Verbatim}

\noindent Note: \cmd{git remote rm} does not delete the remote
repository from the server.  It simply removes the remote and
its references from your local repository.

