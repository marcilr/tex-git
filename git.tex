%% -*- Mode: LaTeX -*-
%%
%% template.tex
%% Created Wed Jul 25 09:32:40 AKDT 2007
%% by Raymond E. Marcil <marcilr@gmail.com>
%% 
%% Git is a free and open source distributed version control system designed
%% to handle everything from small to very large projects with speed and
%% efficiency.
%%
%% Git is easy to learn and has a tiny footprint with lightning fast
%% performance. It outclasses SCM tools like Subversion, CVS, Perforce,
%% and ClearCase with features like cheap local branching, convenient
%% staging areas, and multiple workflows.
%% https://git-scm.com/
%%

  %%
%%%%%% Preamble.
  %%

%% Specify DVIPS driver used by things like hyperref
\documentclass[12pt,letterpaper,dvips]{article}

%% rcs is the package to display cvs revision info.
%%\usepackage{rcs}
\usepackage{fullpage}
\usepackage{fancyvrb} 
\usepackage{graphicx}
\usepackage{figsize}
\usepackage{calc}

%%
%% enumitem – Control layout of itemize, enumerate, description
%% https://www.ctan.org/pkg/enumitem
%%
%% Allows for use of \bgein{itemize}[leftmargin=0pt] 
%% to lists with 0 left margin.
%%
%% Itemize left margin
%% http://tex.stackexchange.com/questions/170525/itemize-left-margin
%% 
\usepackage{enumitem}%     http://ctan.org/pkg/enumitem


%% caption package for use in justifying table or figure captions
\usepackage{caption}

\usepackage{xspace}
\usepackage{booktabs}
\usepackage[first,bottomafter]{draftcopy}
\usepackage[numbib]{tocbibind}

\usepackage{amssymb}              %% AMS Symbols, used for \checkmark
\usepackage{multicol}

%%
%% Extract SVN metadata for use elsewhere.
%% This information has:
%% o the filename
%% o the revision number
%% o the date and time of the last Subversion co command
%% o name of the user who has done the action
%%
\usepackage{svninfo}
\svnInfo $Id$

%%
%% Hyperref package for embedding URLs for clickable links in PDFs, 
%% also specify PDF attributes here.
%%
%% The pdfborder={0 0 0} is what ellimated the blue box around the url
%% displayed by \href{}{}.
%%
%% The command pdfborder={0 0 1} would display a box with thickness of 1 pt.
%%
%% Hypertext marks in LATEX: a manual for hyperref
%% by Sebastian Rahtz and Heiko Oberdiek - November 2012
%% http://ctan.org/pkg/hyperref 
%% http://mirror.hmc.edu/ctan/macros/latex/contrib/hyperref/doc/manual.html
%%
\usepackage[
colorlinks,
linkcolor=blue,
%%colorlinks=false,
hyperindex=false,
urlcolor=blue,
pdfborder={0 0 0},
pdfauthor={Raymond E. Marcil},
pdftitle={Git, Revision Control},
pdfcreator={ps2pdf},
pdfsubject={git, revision control},
pdfkeywords={git, revision control}
]{hyperref}


%%
%% Extract RCS metadata for use elsewhere.
%% Jason figured this out, very cool.
%%
%%\RCS $Revision: 1.53 $
%%\RCS $Date: 2006/06/26 21:04:55 $


  %%
%%%%%% Customization.
  %%

% On letter paper with 10pt font the Verbatim environment has 65 columns.
% With 12pt font the environment has 62 columns.  Exceeding this will exceed
% the frame and will look ugly.  YHBW.  HAND.
\RecustomVerbatimEnvironment{Verbatim}{Verbatim}{frame=single}

\renewenvironment{description}
                 {\list{}{\labelwidth 0pt \iteminden-\leftmargin
                          \let\labelsep\hsize
                          \let\makelabel\descriptionlabel}}
                 {\endlist}
\renewcommand*\descriptionlabel[1]{\hspace\labelsep\sffamily\bfseries #1}


  %%
%%%%%% Commands.
  %%

\newcommand{\FIXME}[1]{\textsf{[FIXME: #1]}}
\newcommand{\cmd}[1]{\texttt{#1}}


%% Squeeze space above/below captions
\setlength{\abovecaptionskip}{4pt}   % 0.5cm as an example
\setlength{\belowcaptionskip}{4pt}   % 0.5cm as an example


%% Tex really adds a lot of whitespace to itemized 
%% lists so define a new command itemize* with a 
%% lot less whitespace.  Found this in the British
%% Tex faq.
\newenvironment{itemize*}%
  {\begin{itemize}%
    \setlength{\itemsep}{0pt}%
    \setlength{\parsep}{0pt}}%
  {\end{itemize}}

  
%%
%% Tex really adds a lot of whitespace to itemized 
%% lists so define a new command itemize* with a 
%% lot less whitespace.  Found this in the British
%% Tex faq.
%%
%% Tue Jun 23 13:22:04 AKDT 2015
%% =============================
%% Added [leftmargin=0.0mm] to set the left margin=0
%% This requires use of the enumitem package:
%%   \usepackage{enumitem}%     http://ctan.org/pkg/enumitem
%%
%% Itemize left margin
%% http://tex.stackexchange.com/questions/170525/itemize-left-margin
%%
\newenvironment{itemizenoleft*}%
  {\begin{itemize}[leftmargin=15.0pt]%
    \setlength{\itemsep}{0pt}%
    \setlength{\parsep}{0pt}}%
  {\end{itemize}}
  

%%
%% Tex really adds a lot of whitespace to itemized 
%% lists so define a new command enumerate* with a 
%% lot less whitespace.  Created using itemize*
%% pattern.  
%%
  \newenvironment{enumerate*}%
  {\begin{enumerate}%
    \setlength{\itemsep}{0pt}%
    \setlength{\parsep}{0pt}}%
  {\end{enumerate}}


%%
%% Tex really adds a lot of whitespace to itemized 
%% lists so define a new command enumerate* with a 
%% lot less whitespace.  Created using itemize*
%% pattern.  
%%
%% Tue Jun 23 13:22:04 AKDT 2015
%% =============================
%% Added [leftmargin=0.0mm] to set the left margin=0
%% This requires use of the enumitem package:
%%   \usepackage{enumitem}%     http://ctan.org/pkg/enumitem
%%
%% Itemize left margin
%% http://tex.stackexchange.com/questions/170525/itemize-left-margin
%%
\newenvironment{enumeratenoleft*}%
  {\begin{enumerate}[leftmargin=0.0mm]%
    \setlength{\itemsep}{0pt}%
    \setlength{\parsep}{0pt}}%
  {\end{enumerate}}


%% Squeeze space
\renewcommand\floatpagefraction{.9}
\renewcommand\topfraction{.9}
\renewcommand\bottomfraction{.9}
\renewcommand\textfraction{.1}   
\setcounter{totalnumber}{50}
\setcounter{topnumber}{50}
\setcounter{bottomnumber}{50}


  %%
%%%%%% Document.
  %%

\title{Git\\
       Revision Control}

\author{Raymond E. Marcil\\
        \texttt{$<$marcilr@gmail.com$>$}
}

% Display subversion revision and date under author on 1st page.
\date{Revision \svnInfoRevision
      \hspace{2pt}
      (\svnInfoLongDate)}


% Set date to RCS revision date
%%\date{Revision \RCSRevision
%%      \hspace{2pt}
%%      (\RCSDate)}


%%
%% Display SVN (subversion) version data at top right of 1st page,
%% This may be preferable to underneath the author.
%%
%%\rhead{Revision \svnInfoRevision\\
%%\svnInfoLongDate}


\begin{document}

\maketitle

\begin{abstract}
  \noindent Git is a free and open source distributed version control
  system designed to handle everything from small to very large projects
  with speed and efficiency.\\
  \\
  Git is easy to learn and has a tiny footprint with lightning fast
  performance. It outclasses SCM tools like Subversion, CVS, Perforce,
  and ClearCase with features like cheap local branching, convenient
  staging areas, and multiple workflows.\footnote{Git - \href{https://git-scm.com/}{https://git-scm.com/}}
\end{abstract}

\vspace{2.0in}

%% Draw DNR logo and address at bottom of page
%%\begin{figure}[h]
%%        \hspace{0.32in}
%%        \SetFigLayout{1}{1}
%%        \begin{minipage}[b]{0.16\figwidth}
%%                \includegraphics[width=\textwidth]{dnr_bwlogo.eps}
%%        \end{minipage}
%%        \hspace{5pt}
%%        \begin{minipage}[b]{\figwidth}
%%                \bf{Alaska Department of Natural Resources}\\
%%                \small{\sf{Division of Support Services\\
%%                Land Records Information Section\\
%%                550 W. 7th Ave. Suite 706\\
%%                Anchorage, Alaska 99501}}
%%        \end{minipage}
%%\end{figure}

\newpage
\tableofcontents

\newpage
\listoffigures
\listoftables


%% =============== List of Abbreviations ===============
%% =============== List of Abbreviations ===============
\newpage
\setcounter{secnumdepth}{0}
\section{List of Definitions and Abbreviations}
\begin{itemize*}
  \item{\begin{bf}Branch\end{bf}} - \FIXME{Need data}
  \item{\begin{bf}Git\end{bf}} - Quoting Linus: I'm an egotistical bastard,
  and I name all my projects after myself. First 'Linux', now 'Git'.
('git' is British slang for ``pig headed, think they are always correct,
    argumentative'').\footnote{Git FAQ\\
    \href{https://git.wiki.kernel.org/index.php/GitFaq\#Why\_the\_.27Git.27\_name.3F}{https://git.wiki.kernel.org/index.php/GitFaq\#Why\_the\_.27Git.27\_name.3F}}
  \item{\begin{bf}Repo\end{bf}} - ``The multiple repository tool.
    Repo is a tool that we built on top of Git.  Repo helps us
    manage the many Git repositories, does the uploads to our
    revision control system, and automates parts of the Android
    development workflow.  Repo is not meant to replace Git,
    only to make it easier to work with Git in the context of
    Android.  The repo command is an executable Python script
    that you can put anywhere in your path.''\\
    \href{https://code.google.com/p/git-repo/}{https://code.google.com/p/git-repo/}
  \item{\begin{bf}Tag\end{bf}} - \FIXME{Need data}
\end{itemize*}


%% ====================== Introduction ===========================
%% ====================== Introduction ===========================
%% ====================== Introduction ===========================
\newpage
\section{Introduction}

Git is a distributed revision control system with an emphasis on
speed,\footnote{ Torvalds, Linus
  (2005-04-07). ``\href{http://marc.info/?l=linux-kernel&m=111288700902396}{Re:
    Kernel SCM saga...}'' linux-kernel (Mailing list). ``So I'm writing some
  scripts to try to track things a whole lot faster.''} data integrity,\footnote{ Torvalds, Linus (2007-06-10). ``\href{http://marc.info/?l=git&m=118143549107708}{Re: fatal: serious inflate inconsistency}''. git (Mailing list). A brief description of Git's data integrity design goals.} and support for distributed, non-linear workflows.\footnote{Linus Torvalds (2007-05-03). \href{https://www.youtube.com/watch?v=4XpnKHJAok8}{Google tech talk: Linus Torvalds on git}. Event occurs at 02:30. Retrieved 2007-05-16.} Git was initially designed and developed by Linus Torvalds for Linux kernel development in 2005, and has since become one of the most widely adopted version control systems for software development.\footnote{ ``\href{http://ianskerrett.wordpress.com/2014/06/23/eclipse-community-survey-2014-results/}{Eclipse Community Survey 2014 results | Ian Skerrett}''. Ianskerrett.wordpress.com. 2014-06-23. Retrieved 2014-06-23.}\\
\\
\noindent As with most other distributed revision control systems,
and unlike most client–server systems, every Git working directory is a
full-fledged repository with complete history and full version-tracking
capabilities, independent of network access or a central
server.\footnote{Chacon, Scott (24
  December 2014). \href{http://git-scm.com/book/en/v2}{Pro Git} (2nd ed.). New
  York, NY: Apress. pp. 29–30. ISBN 978-1484200773.} Like the Linux kernel,
Git is free software distributed under the terms of the GNU General Public
License version 2.\footnote{Git (software), From Wikipedia, the free
encyclopedia, \href{https://en.wikipedia.org/wiki/Git\_(software)}{https://en.wikipedia.org/wiki/Git\_(software)}}


%% ===================== Command Reference =======================
%% ===================== Command Reference =======================
%% ===================== Command Reference =======================
\newpage
\section{Command Reference}

\begin{table}[htb]
\begin{center}
\begin{tabular}{|p{.15\textwidth}|p{.70\textwidth}|}\hline
Command&Description\\
\hline
\cmd{add}&Add file contents to the index\\
\cmd{apply}&Apply a patch to files and/or to the index\\
\cmd{clone}&Get a complete copy of a repository\\
\cmd{commit}&Record changes to the repository\\
\cmd{pull}&Fetch from and integrate with another repository or a local branch\\
\cmd{push}&Update remote refs along with associated objects\\
\cmd{rebase}&Forward-port local commits to the updated upstream head\\
\cmd{status}&Show the working tree status\\
\hline
\end{tabular}
\caption {Commands}
\label{table:commands}
\end{center}
\end{table}


%% ============================ Add ==============================
%% ============================ Add ==============================
\subsection{Add}
Add file contents to the index
\\
\\
This command updates the index using the current content found in
the working tree, to prepare the content staged for the next
commit.  It typically adds the current content of existing paths
as a whole, but with some options it can also be used to add
content with only part of the changes made to the working tree
files applied, or remove paths that do not exist in the working
tree anymore.\footnote{git-add - Add file contents to the index\\
\href{https://git-scm.com/docs/git-add}{https://git-scm.com/docs/git-add}}


%% ========================= Examples ============================
%% ========================= Examples ============================
\subsubsection{Examples}
Create and add \cmd{.repo/manifests/default.xml} file and
\cmd{.repo/manifests/manifest.xml}
\begin{Verbatim}
 covellite:~/git/.repo/manifests$ git add default.xml
 covellite:~/git/.repo/manifests$ git commit default.xml
 covellite:~/git/.repo/manifests$ cd ..
 covellite:~/git/.repo/$ ln -s manifests/default.xml manifest.xml
 covellite:~/git/.repo/$ git add manifest.xml
 covellite:~/git/.repo/$ git commit
\end{Verbatim}


%% =========================== Clone =============================
%% =========================== Clone =============================
\newpage
\subsection{Clone}
To grab a complete copy of another user's repository, use git clone like this:

\begin{Verbatim}
  $ git clone https://github.com/USERNAME/REPOSITORY.git
  # Clones a repository to your computer
\end{Verbatim}

\noindent When you run \cmd{git clone}, the following actions occur:\\
$>$ A new folder called repo is made\\
$>$ It is initialized as a Git repository\\
$>$ A remote named origin is created, pointing to the URL you cloned from\\
$>$ All of the repository's files and commits are downloaded there\\
$>$ The default branch (usually called master) is checked out\\

\noindent For every branch \cmd{foo} in the remote repository, a corresponding remote-tracking
branch \cmd{refs/remotes/origin/foo} is created in your local repository.  You can
usually abbreviate such remote-tracking branch names to \cmd{origin/foo}.\footnote{Fetching a remote,
\cmd{git clone}, \cmd{git fetch}, \cmd{git merge}, \cmd{git pull},
\href{https://help.github.com/articles/fetching-a-remote/}{https://help.github.com/articles/fetching-a-remote/}}


%% ========================= Examples ============================
%% ========================= Examples ============================
\subsubsection{Examples}
To clone repository named \cmd{git} from GitHub to local \cmd{covellite} workstation:
\begin{Verbatim}
 covellite:~$ git clone https://github.com/marcilr/git.git
 Cloning into 'git'...
 warning: You appear to have cloned an empty repository.
 Checking connectivity... done.
 covellite:~$
\end{Verbatim}

\newpage
\noindent Clone \cmd{.repo} repository into \cmd{git/} directory:
\begin{Verbatim}
 covellite:~/git$ git clone https://github.com/marcilr/.repo
 Cloning into '.repo'...
 warning: You appear to have cloned an empty repository.
 Checking connectivity... done.
 covellite:~/git$
\end{Verbatim}


\newpage
\noindent To clone a Git repository over SSH, you can specify ssh:// URL like this:\\
\begin{Verbatim}
 $ git clone ssh://user@server/project.git
\end{Verbatim}

\noindent Or you can use the shorter scp-like syntax for the SSH protocol:

\begin{Verbatim}
 $ git clone user@server:project.git
\end{Verbatim}

\noindent You can also not specify a user, and Git assumes the user
you're currently logged in as.\footnote{
Git on the Server - The Protocols, The SSH Protocol,\\
\href{https://git-scm.com/book/en/v2/Git-on-the-Server-The-Protocols}{https://git-scm.com/book/en/v2/Git-on-the-Server-The-Protocols}}
\\
\\
\noindent \FIXME{Need more commands here.}


%% ========================= Remotes =============================
%% ========================= Remotes =============================
\newpage
\subsection{Remotes}
``To be able to collaborate on any Git project, you need to know
how to manage your remote repositories.  Remote repositories are
versions of your project that are hosted on the Internet or
network somewhere.  You can have several of them, each of which
generally is either read-only or read/write for you.
Collaborating with others involves managing these remote
repositories and pushing and pulling data to and from them when
you need to share work.  Managing remote repositories includes
knowing how to add remote repositories, remove remotes that are no
longer valid, manage various remote branches and define them as
being tracked or not, and more. In this section, we’ll cover some
of these remote-management skills.''\footnote{Git Basics - Working
with Remotes,\\
\href{http://git-scm.com/book/en/v2/Git-Basics-Working-with-Remotes}{http://git-scm.com/book/en/v2/Git-Basics-Working-with-Remotes}}


%% ================== Remote Repositories ========================
%% ================== Remote Repositories ========================
\subsubsection{Remote Repositories}
GitHub's collaborative approach to development depends on
publishing commits from your local repository for other people to
view, fetch, and update.\footnote{About remote repositories,
\href{https://help.github.com/articles/about-remote-repositories/}{https://help.github.com/articles/about-remote-repositories/}}
\\
\\
A remote URL is Git's fancy way of saying ``the place where your
code is stored.'' That URL could be your repository on GitHub, or
another user's fork, or even on a completely different server.
\\
\\
You can only push to two types of URL addresses:
\\
\\
$>$ An HTTPS URL like \cmd{https://github.com/user/repo.git}\\
$>$ An SSH URL, like \cmd{git@github.com:user/repo.git}\\
\\
Git associates a remote URL with a name, and your default remote is usually called origin.


%% ===================== Remote Commands =========================
%% ===================== Remote Commands =========================
\newpage
\subsubsection{Remote Commands}
\begin{table}[htb]
\begin{center}
\begin{tabular}{|p{.20\textwidth}|p{.70\textwidth}|}\hline
Command&Description\\
\hline
\cmd{remote add}&Add a new remote in the directory your repository is stored at.\\
\cmd{remote set-url}&Change a remote's URL.\\
\cmd{remote rename}&Rename an existing remote.\\
\cmd{remote rm}&Remove a remote URL from your repository.\\
\hline
\end{tabular}
\caption {Remote Commands}
\label{table:remote_commands}
\end{center}
\end{table}

\noindent For further reading see ``\href{http://git-scm.com/book/en/Git-Basics-Working-with-Remotes}{Working with Remotes}''
from the Pro Git book.\footnote{Pro Git book, \href{http://git-scm.com/book/en/Git-Basics-Working-with-Remotes}{http://git-scm.com/book/en/Git-Basics-Working-with-Remotes}}


%% ===================== git remote add =========================
%% ===================== git remote add =========================
\newpage
\subsubsection{git remote add}
To add a new remote, use the \cmd{git remote add} command on the
terminal, in the directory your repository is stored at.\footnote{Adding a remote,
\href{https://help.github.com/articles/adding-a-remote/}{https://help.github.com/articles/adding-a-remote/}}
\\
\\
The \cmd{git remote add} command takes two arguments:
\vspace{-10pt}
\begin{verbatim}
$ git remote add <NAME> <REMOTE_URL> 
\end{verbatim}

\noindent Where:
\vspace{-10pt}
\begin{table}[htb]
%%\begin{table}
\begin{center}
\begin{tabular}{p{.20\textwidth}p{.80\textwidth}}
\hspace{20pt}\cmd{<NAME>}&- A remote name, for example, \cmd{origin}\\
\hspace{20pt}\cmd{<REMOTE\_URL>}&- A remote URL, for example, \cmd{https://github.com/user/repo.git}\\
\end{tabular}
\label{table:remote_add}
\end{center}
\end{table}

\vspace{-20pt}
\noindent Git associates a remote URL with a name, and your
default remote is usually called origin.\footnote{About remote repositories,
\href{https://help.github.com/articles/about-remote-repositories/}{https://help.github.com/articles/about-remote-repositories/}}
\\

%% ======================== Examples ============================
%% ======================== Examples ============================
\vspace{10pt}
\noindent \begin{bf}Examples\end{bf}
\begin{Verbatim}
 # This associates the name origin with SSH URL for repo.git repository.
 $ git remote add origin git@github.com:user/repo.git
\end{Verbatim}

\noindent Alternatively using https syntax:

\begin{Verbatim}
 $ git remote add origin https://github.com/user/repo.git
 # Set a new remote

 $ git remote -v
 # Verify new remote
 origin  https://github.com/user/repo.git (fetch)
 origin  https://github.com/user/repo.git (push)
\end{Verbatim}


%% =================== git remote set-url =======================
%% =================== git remote set-url =======================
\newpage
\subsubsection{git remote set-url}
The \cmd{git remote set-url} command changes an existing remote
repository URL.\footnote{Changing a remote's URL,
\href{https://help.github.com/articles/changing-a-remote-s-url/}{https://help.github.com/articles/changing-a-remote-s-url/}}
\\
\\
The git remote set-url command takes two arguments:
\\
$>$ An existing remote name. For example, \cmd{origin} or \cmd{upstream} are two common choices.\\
$>$ A new URL for the remote. For example:\\

\noindent\hspace*{10pt}$>$ If you're updating to use HTTPS, your URL might look like:\\
\noindent\hspace*{23pt}\cmd{https://github.com/USERNAME/OTHERREPOSITORY.git}\\
\\
\noindent\hspace*{10pt}$>$ If you're updating to use SSH, your URL might look like:\\
\noindent\hspace*{23pt}\cmd{git@github.com:USERNAME/OTHERREPOSITORY.git}


%% ========= Switching remote URLs from SSH to HTTPS ===========
%% ========= Switching remote URLs from SSH to HTTPS ===========
\subsubsection{Switching remote URLs from SSH to HTTPS}
\begin{enumerate}
  \item{Open Terminal (for Mac and Linux users) or the command prompt
       (for Windows users).\footnote{Switching remote URLs from HTTPS to SSH,
       \href{https://help.github.com/articles/changing-a-remote-s-url/}{https://help.github.com/articles/changing-a-remote-s-url/}}}
  \item{Change the current working directory to your local project.}
  \item{List your existing remotes in order to get the name of the remote you want to change.
\begin{Verbatim}
 $ git remote -v
 # origin  git@github.com:USERNAME/REPOSITORY.git (fetch)
 # origin  git@github.com:USERNAME/REPOSITORY.git (push)
\end{Verbatim}
       }
  \item{Change your remote's URL from SSH to HTTPS with the git remote set-url command.
\begin{Verbatim}
 $ git remote set-url origin \
 https://github.com/USERNAME/OTHERREPOSITORY.git
\end{Verbatim}
      }

  \item{Verify that the remote URL has changed.
\begin{Verbatim}
$ git remote -v
# Verify new remote URL
# origin  https://github.com/USERNAME/OTHERREPOSITORY.git (fetch)
# origin  https://github.com/USERNAME/OTHERREPOSITORY.git (push)
\end{Verbatim}
       }
\end{enumerate}


%% ========= Switching remote URLs from HTTPS to SSH ===========
%% ========= Switching remote URLs from HTTPS to SSH ===========
\newpage
\subsubsection{Switching remote URLs from HTTPS to SSH}
\begin{enumerate}
  \item{Open Terminal (for Mac and Linux users) or the command prompt
       (for Windows users).\footnote{Switching remote URLs from HTTPS to SSH,
       \href{https://help.github.com/articles/changing-a-remote-s-url/}{https://help.github.com/articles/changing-a-remote-s-url/}}}
  \item{Change the current working directory to your local project.}
  \item{List your existing remotes in order to get the name of the remote you want to change.
\begin{Verbatim}
 $ git remote -v
 origin  https://github.com/USERNAME/REPOSITORY.git (fetch)
 origin  https://github.com/USERNAME/REPOSITORY.git (push)
\end{Verbatim}
       }
  \item{Change your remote's URL from HTTPS to SSH with the git remote set-url command.
\begin{Verbatim}
 $ git remote set-url origin \
 git@github.com:USERNAME/OTHERREPOSITORY.git
\end{Verbatim}
      }

  \item{Verify that the remote URL has changed.
\begin{Verbatim}
 $ git remote -v
 # Verify new remote URL
 origin  git@github.com:USERNAME/OTHERREPOSITORY.git (fetch)
 origin  git@github.com:USERNAME/OTHERREPOSITORY.git (push)
\end{Verbatim}
       }
\end{enumerate}


%% ==================== git remote rename ======================
%% ==================== git remote rename ======================
\subsubsection{git remote rename}
Use the git remote rename command to rename an existing remote.\footnote{Renaming a remote,
\href{https://help.github.com/articles/renaming-a-remote/}{https://help.github.com/articles/renaming-a-remote/}}
\\
\\
The git remote rename command takes two arguments:
\\
\\
$>$ An existing remote name, for example, origin A new name for the\\
$>$ remote, for example, destination

\newpage
\noindent \begin{bf}Example\end{bf}\\
The examples below assume you're cloning using HTTPS, which is recommended.

\begin{Verbatim}
 $ git remote -v
 # View existing remotes
 origin  https://github.com/OWNER/REPOSITORY.git (fetch)
 origin  https://github.com/OWNER/REPOSITORY.git (push)

 $ git remote rename origin destination
 # Change remote name from 'origin' to 'destination'

 $ git remote -v
 # Verify remote's new name
 destination  https://github.com/OWNER/REPOSITORY.git (fetch)
 destination  https://github.com/OWNER/REPOSITORY.git (push)
\end{Verbatim}


%% ====================== git remote rm ======================
%% ====================== git remote rm ======================
\subsubsection{git remote rm}
Use the \cmd{git remote rm} command to remove a remote URL
from your repository.\footnote{Removing a remote,
\href{https://help.github.com/articles/removing-a-remote/}{https://help.github.com/articles/removing-a-remote/}}
\\
\\
The \cmd{git remote rm} command takes one argument:
\\
\\
$>$ A remote name, for example, destination
\\
\\
\noindent \begin{bf}Example\end{bf}
\\
\\
The examples below assume you're cloning using HTTPS, which is recommended.

\begin{Verbatim}
 $ git remote -v
 # View current remotes
 origin  https://github.com/OWNER/REPOSITORY.git (fetch)
 origin  https://github.com/OWNER/REPOSITORY.git (push)
 destination  https://github.com/FORKER/REPOSITORY.git (fetch)
 destination  https://github.com/FORKER/REPOSITORY.git (push)

 $ git remote rm destination
 # Remove remote
 $ git remote -v
 # Verify it's gone
 origin  https://github.com/OWNER/REPOSITORY.git (fetch)
 origin  https://github.com/OWNER/REPOSITORY.git (push)
\end{Verbatim}

\noindent Note: \cmd{git remote rm} does not delete the remote
repository from the server.  It simply removes the remote and
its references from your local repository.


%% ==================== The SSH Protocol =========================
%% ==================== The SSH Protocol =========================
\newpage
\section{The SSH Protocol}
A common transport protocol for Git when self-hosting is over SSH.
This is because SSH access to servers is already set up in most
places – and if it isn’t, it's easy to do.  SSH is also an
authenticated network protocol; and because it’s ubiquitous,
it's generally easy to set up and use.
\\
\\
If you have two-factor authentication\footnote{About Two-Factor Authentication,
\href{https://help.github.com/articles/about-two-factor-authentication/}{https://help.github.com/articles/about-two-factor-authentication/}}
enabled, you must create a personal access token\footnote{Creating
an access token for command-line use to use instead of your GitHub password.,
\href{https://help.github.com/articles/creating-an-access-token-for-command-line-use/}{https://help.github.com/articles/creating-an-access-token-for-command-line-use/}}
 to use instead of your GitHub password.\footnote{Changing a remote's URL,
 \href{https://help.github.com/articles/changing-a-remote-s-url/}{https://help.github.com/articles/changing-a-remote-s-url/}}
\\
\\
To clone a Git repository over SSH, you can specify ssh:// URL like this:

\begin{Verbatim}
 $ git clone ssh://user@server/project.git
\end{Verbatim}

\noindent Or you can use the shorter scp-like syntax for the SSH
protocol:

\begin{Verbatim}
 $ git clone user@server:project.git
\end{Verbatim}

\noindent You can also not specify a user, and Git assumes the
user you’re currently logged in as.


%% ======================== The Pros =============================
%% ======================== The Pros =============================
\subsubsection{The Pros}
The pros of using SSH are many.  First, SSH is relatively easy
to set up – SSH daemons are commonplace, many network admins have
experience with them, and many OS distributions are set up with
them or have tools to manage them.  Next, access over SSH is
secure – all data transfer is encrypted and authenticated.  Last,
like the HTTP/S, Git and Local protocols, SSH is efficient, making
the data as compact as possible before transferring it.


%% ======================== The Cons =============================
%% ======================== The Cons =============================
\subsubsection{The Cons}
The negative aspect of SSH is that you can’t serve anonymous
access of your repository over it.  People must have access to
your machine over SSH to access it, even in a read-only capacity,
which doesn’t make SSH access conducive to open source projects.
If you’re using it only within your corporate network, SSH may be
the only protocol you need to deal with.  If you want to allow
anonymous read-only access to your projects and also want to use
SSH, you’ll have to set up SSH for you to push over but something
else for others to fetch over.\footnote{Ibid.}


%% =================== Branching & Tagging =======================
%% =================== Branching & Tagging =======================
\newpage
\section{Branching \& Tagging}

In short: Best practice is branch out, merge often and keep
always in sync.\\
\\
\noindent There are pretty clear conventions about keeping your
code in a separate branches from master branch:

\begin{enumerate*}
\item You are about to make an implementation of major or
disruptive change
\item You are about to make some changes that might not be used
\item You want to experiment on something that you are not sure
    it will work
\item When you are told to branch out, others might have
    something they need to do in master
\end{enumerate*}

\noindent Rule of thumb is after branching out, you should keep in sync with
the master branch. Because eventually you need to merge it back to master. In
order to avoid a huge complicated mess of conflicts when merging back, you
should commit often, merge often.\footnote{Git branching and tagging best
practices\\
\href{http://programmers.stackexchange.com/questions/165725/git-branching-and-tagging-best-practices}{http://programmers.stackexchange.com/questions/165725/git-branching-and-tagging-best-practices}}


%% ==================== Cloud Repository ===========================
%% ==================== Cloud Repositoy ===========================
\newpage
\section{Cloud Repository}
A cloud repository provides easy access from distributed
locations and alleviates backup issues.  Candidates for
a cloud repository include Bitbucket,\footnote{Bitbucket - Code, Manage, Collaborate,
Bitbucket is the Git solution for professional teams\\
\href{https://bitbucket.org/}{https://bitbucket.org/}}
GitHub,\footnote{GitHub - Where software is built\\
\href{https://github.com/}{https://github.com/}}
or Google Code.\footnote{Google Code - Provides a free collaborative development environment
for open source projects.\\
\href{https://code.google.com/}{https://code.google.com/}}


%% ========================= GitHub ==============================
%% ========================= GitHub ==============================
\subsection{GitHub}

\FIXME{Still need cli list, rename, and delete functionality.}


%% ========= Caching your GitHub password in Git =================
%% ========= Caching your GitHub password in Git =================
\subsubsection{Caching your GitHub password in Git}
If you're cloning GitHub repositories using HTTPS, you can use a
credential helper to tell Git to remember your GitHub username
and password every time it talks to GitHub.
\\
\\
If you clone GitHub repositories using SSH, then you authenticate
using SSH keys instead of a username and password.  For help
setting up an SSH connection, see Generating SSH Keys.\footnote{Generating SSH keys,
\href{https://help.github.com/articles/generating-ssh-keys/}{https://help.github.com/articles/generating-ssh-keys/}}
\\
\\
Turn on the credential helper so that Git will save your password
in memory for some time.  By default, Git will cache your password
for 15 minutes.\footnote{Caching your GitHub password in Git,
\href{https://help.github.com/articles/caching-your-github-password-in-git/}{https://help.github.com/articles/caching-your-github-password-in-git/}}

\begin{enumerate}
  \item{In Terminal, enter the following:
\begin{Verbatim}
 $ git config --global credential.helper cache
 # Set git to use the credential memory cache
\end{Verbatim}
       }
  \item{To change the default password cache timeout, enter the following:
\begin{Verbatim}
 $ git config --global credential.helper 'cache --timeout=3600'
 # Set the cache to timeout after 1 hour (setting is in seconds)
\end{Verbatim}
       }
\end{enumerate}

\noindent \FIXME{The GitHub user and pass got saved by alternate means.  How was that done?}

\subsubsection{Create a Github Repo from the Command Line}
Creating a GitHub repository from the command line is incredibly
convenient.
\\
\\
Googled up some simple shell script to create GitHub repo via
command line:

\begin{verbatim}
  "curl -u $username:$token" https://api.github.com/user/repos \
  -d '{"name":"'$repo_name'"}'}
\end{verbatim}

\noindent To use, you could simply replace \cmd{\$username} with your
GitHub username, \cmd{\$token} with a Personal Access Token\footnote{GitHub
supports Personal access tokens, under Settings, click Personal access
tokens. Personal access tokens function like ordinary OAuth access tokens.
They can be used instead of a password for Git over HTTPS, or can be used
to authenticate to the API over Basic Authentication.  I set mine to the usual:)\\
\href{https://github.com/settings/tokens}{https://github.com/settings/tokens}}
for the same user (available for generation in your GitHub\\
Settings $>$ Applications), and \cmd{\$repo\_name} with your
desired new Repository name.\footnote{Create a Github Repo
from the Command Line, by Eli Fatsi - Jan 29, 2014,\\
\href{https://viget.com/extend/create-a-github-repo-from-the-command-line}{https://viget.com/extend/create-a-github-repo-from-the-command-line}}
\\
\\
Creating a repo from the command line is definitely faster
than going to Github and using the web app to get the job
done, but in order to truly make this task speedy, we need
some Bash programming.

\newpage
\begin{Verbatim}
github-create() {
  repo_name=$1
 
  dir_name=`basename $(pwd)`
 
  if [ "$repo_name" = "" ]; then
    echo "Repo name (hit enter to use '$dir_name')?"
    read repo_name
  fi
 
  if [ "$repo_name" = "" ]; then
    repo_name=$dir_name
  fi
 
  username=`git config github.user`
  if [ "$username" = "" ]; then
    echo "Could not find username, run 'git config \
    --global github.user <username>'"
    invalid_credentials=1
  fi
 
  token=`git config github.token`
  if [ "$token" = "" ]; then
    echo "Could not find token, run 'git config \
    --global github.token <token>'"
    invalid_credentials=1
  fi
 
  if [ "$invalid_credentials" == "1" ]; then
    return 1
  fi
 
  echo -n "Creating Github repository '$repo_name' ..."
  curl -u "$username:$token" https://api.github.com/user/repos \
  -d '{"name":"'$repo_name'"}' > /dev/null 2>&1
  echo " done."
 
  echo -n "Pushing local code to remote ..."
  git remote add origin git@github.com:$username/$repo_name.git \
  > /dev/null 2>&1
  git push -u origin master > /dev/null 2>&1
  echo " done."
}
\end{Verbatim}

\noindent Plop this function into your \cmd{\textasciitilde/.bash\_profile}, open a new
Terminal window or source \cmd{\textasciitilde/.bash\_profile}, and the function
will be loaded up and ready for use.
\\
\\
Then while in an existing git project, running \cmd{github-create}
will create the repo and push your master branch up in one shot.
You will need to set some github config variables (instructions
will be spit out if you don't have them).  Here’s an example:
\\
\\
\begin{Verbatim}
 BASH:projects $ rails new my_new_project
   ..... (a whole lot of generated things)
 BASH:projects $ cd my_new_project/
 BASH:my_new_project $ git init && git add . && git commit \
 -m 'Initial commit'
  ..... (a whole lot of git additions)
 BASH:my_new_project $ github-create
   Repo name (hit enter to use 'my_new_project')?
 
   Creating Github repository 'my_new_project' ... done.
   Pushing local code to remote ... done.
\end{Verbatim}

\noindent Had I called the function with an argument — \cmd{github-create my\_project} —
then it would have used the argument and skipped the Repo name question.\footnote{Create a Github Repo
from the Command Line, by Eli Fatsi - Jan 29, 2014,\\
\href{https://viget.com/extend/create-a-github-repo-from-the-command-line}{https://viget.com/extend/create-a-github-repo-from-the-command-line}}
\\
\\
On GCI Network Services, OSS \cmd{covellite} Debian jessie 8.2
workstation the \cmd{gtihub-create} did not execute.  Put
github-create() into a standalone \cmd{\textasciitilde/gtihub-create}
script with call to \cmd{gtihub-create()} under main.
\\
\\
Tested with:
\begin{Verbatim}
 $ mkdir ~/quux
 $ cd ~/quux
 $ git init
 Initialized empty Git repository in /home/marcilr/quux/.git/
 $ github-create 
 Repo name (hit enter to use 'quux')?

 Could not find username, run 'git config --global github.user <username>'
 Could not find token, run 'git config --global github.token <token>'
 $
\end{Verbatim}

\newpage
\noindent Configured the GetHub username and token to alleviate GutHub credential errors:
\begin{Verbatim}
 $ git config --global github.user marcilr
 $ git config --global github.token <token>
\end{Verbatim}

\noindent Was then able to run \cmd{\textasciitilde/github-create} successfully:
\begin{Verbatim}
 $ cd ~/quux/
 $ github-create
 Repo name (hit enter to use 'quux')? <enter>

 Creating Github repository 'quux' ... done.
 Pushing local code to remote ... done.
 $
\end{Verbatim}

\noindent Checking GtiHub via online access I found the new \cmd{quux} repo.
\\
\\
\noindent \FIXME{Need Bitbucket vs. GitHub section}


%% Adding an existing project to GitHub using the command line
%% https://help.github.com/articles/adding-an-existing-project-to-github-using-the-command-line/

%% Adding a file to a repository from the command line
%% https://help.github.com/articles/adding-a-file-to-a-repository-from-the-command-line/


%% ========================== Repo ===============================
%% ========================== Repo ===============================
%% ========================== Repo ===============================
\newpage
\section{Repo}
``Repo is a repository management tool that we built on top of Git.
Repo unifies the many Git repositories when necessary, does the
uploads to a revision control system, and automates parts of the
development workflow.  Repo is not meant to replace Git, only to
make it easier to work with Git in the context of Android.  The
\cmd{repo} command is an executable Python script that you can put
anywhere in your path.  In working with source files, you will use
Repo for across-network operations.  For example, with a single
Repo command you can download files from multiple repositories
into your local working directory.''\footnote{Developing -- \href{http://source.android.com/source/developing.html}{http://source.android.com/source/developing.html}}
\\
\\
\noindent \FIXME{The above \cmd{repo} quote has been heavily modified.
Need to rewrite with original verbage.}


%% ================== .repo/ subdirectory ===========================
%% ================== .repo/ subdirectory ===========================
\subsection{\cmd{.repo/} subdirectory}
The \cmd{.repo/} subdirectory, located in the repository base,
holds repo configuration.  The configuration includes a manifest
with information about all the projects and where their associated git
repositories are located.
\\
\\
Files within the \cmd{.repo/} subdirectory includes:
\begin{verbatim}
  manifests/
  manifests.git
  manifest.xml -> manifests/default.xml
  project-objects
  projects/
  repo/
\end{verbatim}
 
\noindent To create the \cmd{.repo/} subdirectory:

\begin{Verbatim}
 $ cd <my_repo>
 $ mkdir .repo/
 $ 
\end{Verbatim}

%% ======================== Manifest =============================
%% ======================== Manifest =============================
\subsection{Manifest}
The repo keeps a manifest, ``within the hidden directory named `.repo','' 
in ``a git project named `\cmd{manifests}' which
usually contains a file named `\cmd{default.xml}'.  This file
contains information about all the projects and where their associated
git repositories are located.  This file is also versioned thus when
you use the `\cmd{repo init -b XYZ}' command it will be reverted and you
can back to older branches that may have added/removed git projects
compared to the head.''\footnote{How does the Android repo manifest repository work?\\
\href{http://stackoverflow.com/questions/6149725/how-does-the-android-repo-manifest-repository-work}
{http://stackoverflow.com/questions/6149725/how-does-the-android-repo-manifest-repository-work}}
\\
\\
The \cmd{default.xml} file is symlinked to \cmd{.repo/manifest.xml} and
is created when the repo was initialized using:
\\
\\
\cmd{repo init -u $<$manifest path$>$}


%% ======================== Examples =============================
%% ======================== Examples =============================
\subsubsection{Examples}
Following is a manifest, in \cmd{.repo/manifests/default.xml} file, showing use
of GitHub with username, ssh:// URL syntax, and 3 project repos
with different usernames:\footnote{Keiji Ariyama, \href{https://github.com/keiji/repo-sample/blob/master/default.xml}{https://github.com/keiji/repo-sample/blob/master/default.xml}}

\begin{Verbatim}
 <?xml version="1.0" encoding="UTF-8"?>
 <manifest>
   <remote name="origin" fetch="ssh://git@github.com/" />
   <default revision="master" remote="origin" />

   <project path="lib/plist-lib"
            name="keiji/AndroidPListLib.git" remote="origin" />

   <project path="lib/json-pull-parser"
            name="vvakame/JsonPullParser.git" remote="origin" />

   <project path="apps/twicca_megane_plugin"
            name="zaki50/TwiccaMeganePlugin.git" remote="origin" />
 </manifest>
\end{Verbatim}


%% ======================== Commands =============================
%% ======================== Commands =============================
\newpage
\subsection{Commands}
Repo usage takes the following form:\footnote{Repo command reference\\
\href{https://source.android.com/source/using-repo.html\#help}{https://source.android.com/source/using-repo.html\#help}}
\\
\indent \cmd{repo <COMMAND> <OPTIONS>}
\\
\\
Optional elements are shown in brackets [ ].  For example, many commands take
a project list as an argument.  You can specify project-list as a list of
names or a list of paths to local source directories for the projects:

\indent \cmd{repo sync [<PROJECT0> <PROJECT1> <PROJECTN>]}\\
\indent \cmd{repo sync [</PATH/TO/PROJECT0> ... </PATH/TO/PROJECTN>]}
\\

\noindent Once Repo is installed, you can find the latest documentation
starting with a summary of all commands by running:
\\
\indent \cmd{repo help}
\\
\\
\noindent You can get information about any command by running this within a Repo tree:
\\
\indent \cmd{repo help <COMMAND>}
\\
\\
NOTE: For \cmd{repo} commands without syntax here see the 
Repo command reference.\footnote{Ibid.}


\begin{table}[htb]
\begin{center}
\begin{tabular}{|p{.25\textwidth}|p{.47\textwidth}|}\hline
Command&Description\\
\hline
abandon&Permanently abandon a development branch\\
branch&View current topic branches\\
branches&View current topic branches\\
checkout&Checkout a branch for development\\
cherry-pick&Cherry-pick a change\\
diff&Show changes between commit and working tree\\
diffmanifests&Manifest diff utility\\
download&Download and checkout a change\\
grep&Print lines matching a pattern\\
forall&Executes the given shell command in each project.\footnote{Repo command reference --
\href{https://source.android.com/source/using-repo.html\#forall}{https://source.android.com/source/using-repo.html\#forall}}\\
help&Display detailed help on a command\\
info&Get info on the manifest branch, current branch or unmerged branches\\
init&Install repo in the current working directory\\
list&List projects and their associated directories\\
overview&Display overview of unmerged project branches\\
prune&Prune (delete) already merged topics\\
rebase&Rebase local branches on upstream branch\\
start& Start a new branch for development\\
status&Show the working tree status\\
sync&Update working tree to the latest revision\\
upload&Upload changes for code review\\
\hline
\end{tabular}
\caption {Repo Commands}
\label{table:commands}
\end{center}
\end{table}



%% ========================== init ===============================
%% ========================== init ===============================
\clearpage
\subsubsection{init}
\cmd{\$ repo init -u <URL> [<OPTIONS>]}
\\
Installs Repo in the current directory. This creates a \cmd{.repo/}
directory that contains Git repositories for the Repo source code
and the standard Android manifest files.  The \cmd{.repo/}
directory also contains \cmd{manifest.xml}, which is a symlink to the
selected manifest in the \cmd{.repo/manifests/} directory.\footnote{Repo command reference --
\href{https://source.android.com/source/using-repo.html\#forall}{https://source.android.com/source/using-repo.html}}\\

\begin{table}[htb]
\begin{center}
\begin{tabular}{|p{.15\textwidth}|p{.75\textwidth}|}\hline
\centering Command&\centering Description\tabularnewline
\hline
\centering -u&Specify a URL from which to retrieve a manifest repository.
The common manifest can be found at:
\cmd{https://android.googlesource.com/platform/manifest}\\
\centering -m&Select a manifest file within the repository. If no manifest
 name is selected, the default is default.xml.\\
\centering -b&Specify a revision, i.e., a particular manifest-branch.\\
\hline
\end{tabular}
\caption {init Options}
\label{table:init_options}
\end{center}
\end{table}

%% ======================== Examples =============================
%% ======================== Examples =============================
\subsubsection{Examples}
This will create a new place to hold your local copy of the source
tree.   ``url'' should point to a Manifest repository that
describes the whole sources.  It is a special project with a file
(default.xml) that lists all the projects that Android is made of.
In the Manifest file, each projects has attributes about: where to
place it in the tree, where to download it from (git server),
revision that will be used (usually a branch name, tag or commit
sha-id).\footnote{Repo: Tips \& Tricks,\\
\href{http://xda-university.com/as-a-developer/repo-tips-tricks}{http://xda-university.com/as-a-developer/repo-tips-tricks}}
\\
\\
\indent\cmd{repo init -u <url> -b <branch>}
\\
\\
\noindent Note: For all remaining Repo commands, the current working
directory must either be the parent directory of \cmd{.repo/}
or a subdirectory of the parent directory.\footnote{Repo command reference --
\href{https://source.android.com/source/using-repo.html\#forall}{https://source.android.com/source/using-repo.html}}\\
\\
\\
\noindent \FIXME{Need example of GitHub checkout}


%% ======================== Appendix =============================
%% ======================== Appendix =============================
%% ======================== Appendix =============================
%%
%% This will add a standard non-numbered Appendix to the document.
%% The next section Appendix chnages secnumdepth such that the Appendix
%% is not numbered but still displayed in the table of contents (TOC).
%%
%% Adding unnumbered sections to TOC
%% http://tex.stackexchange.com/questions/11668/adding-unnumbered-sections-to-toc
%% 
%% \section*{Appendix}
%% Need content here.
%%
\newpage
\setcounter{secnumdepth}{0}
\section{Appendix}


%% ============= Links ===============
%% ============= Links ===============
A successful Git branching model\\
by Vincent Driessen on Tuesday, January 05, 2010\\
Fine branching diagram here.\\
\href{http://nvie.com/posts/a-successful-git-branching-model/}{http://nvie.com/posts/a-successful-git-branching-model/}
\\
\\
Bitbucket vs. GitHub: Which project host has the most?\\
The right choice boils down to a number of factors -- you might even consider using both\\
\href{http://www.infoworld.com/article/2611771/application-development/application-development-bitbucket-vs-github-which-project-host-has-the-most.html}{http://www.infoworld.com/article/2611771/application-development/application-development-bitbucket-vs-github-which-project-host-has-the-most.html}
\\
\\
Developing\\
Has Repo and Gerrit details with syntax and examples.\\
\href{http://source.android.com/source/developing.html}{http://source.android.com/source/developing.html}
\\
\\
Fetching a remote\\
$>$ \cmd{git clone}\\
$>$ \cmd{git fetch}\\
$>$ \cmd{git merge}\\
$>$ \cmd{git pull}\\
\href{https://help.github.com/articles/fetching-a-remote/}{https://help.github.com/articles/fetching-a-remote/}
\\
\\
Git\\
\href{https://git-scm.com/}{https://git-scm.com/}
\\
\\
Git (software)\\
From Wikipedia, the free encyclopedia\\
\href{https://en.wikipedia.org/wiki/Git\_(software)}{https://en.wikipedia.org/wiki/Git\_(software)}
\\
\\
Git About\\
\href{https://git-scm.com/about}{https://git-scm.com/about}
\\
\\
Git branching and tagging best practices\\
Excellent details and semantics.\\
\href{http://programmers.stackexchange.com/questions/165725/git-branching-and-tagging-best-practices}{http://programmers.stackexchange.com/questions/165725/git-branching-and-tagging-best-practices}
\\
\\
Git FAQ\\
\href{https://git.wiki.kernel.org/index.php/GitFaq}{https://git.wiki.kernel.org/index.php/GitFaq}
\\
\\
Git on the Server - The Protocols, The SSH Protocol\\
The Git Book\\
\href{https://git-scm.com/book/en/v2/Git-on-the-Server-The-Protocols}{https://git-scm.com/book/en/v2/Git-on-the-Server-The-Protocols}\\
\\
\\
Git (software)\\
From Wikipedia, the free encyclopedia\\
\href{https://en.wikipedia.org/wiki/Git\_(software)}{https://en.wikipedia.org/wiki/Git\_(software)}
\\
\\
\noindent Git repositories on gerrit\\
\href{https://gerrit.googlesource.com/}{https://gerrit.googlesource.com/}
\\
\\
GitHub\\
Project host\\
\href{https://github.com/}{https://github.com/}
\\
\\
How does the Android repo manifest repository work?\\
\href{http://stackoverflow.com/questions/6149725/how-does-the-android-repo-manifest-repository-work}
{http://stackoverflow.com/questions/6149725/how-does-the-android-repo-manifest-repository-work}
\\
\\
Installing Repo\\
\href{http://source.android.com/source/downloading.html\#installing-repo}{http://source.android.com/source/downloading.html\#installing-repo}
\\
\\
Managing Remotes\\
\href{https://help.github.com/categories/managing-remotes/}{https://help.github.com/categories/managing-remotes/}
\\
\\
Manifest Format for repo\\
\href{https://gerrit.googlesource.com/git-repo/+/master/docs/manifest-format.txt}{https://gerrit.googlesource.com/git-repo/+/master/docs/manifest-format.txt}
\\
\\
\noindent Pro Git (the git book)\\
Available as
  \href{https://progit2.s3.amazonaws.com/en/2015-08-09-23511/progit-en.661.pdf}{pdf},
  \href{https://progit2.s3.amazonaws.com/en/2015-08-09-23511/progit-en.661.epub}{epub},
  \href{https://progit2.s3.amazonaws.com/en/2015-08-09-23511/progit-en.661.mobi}{mobi},
  and \href{https://progit2.s3.amazonaws.com/en/2015-08-09-23511/progit-en.661.zip}{html}.\\
\href{http://git-scm.com/book/en/v2/}{http://git-scm.com/book/en/v2/}
\\
\\
Re: repo + private repositories in github\\
Details on manifest for google repo use.\\
\href{https://groups.google.com/forum/embed/\#!topic/repo-discuss/kCXO-NdFvj4}{https://groups.google.com/forum/embed/\#!topic/repo-discuss/kCXO-NdFvj4}
\\
\\
Repo Command Reference\\
Using Repo and Git - very useful details here.\\
\href{http://source.android.com/source/using-repo.html}{http://source.android.com/source/using-repo.html}
\\
\\
Repo: Tips \& Tricks\\
\href{http://xda-university.com/as-a-developer/repo-tips-tricks}{http://xda-university.com/as-a-developer/repo-tips-tricks}
\\
\\
repo - The multiple repository tool\\
\href{https://code.google.com/p/git-repo/}{https://code.google.com/p/git-repo/}
\\
\\
Set Up Git\\
$>$Creating a repository\\
$>$Forking a repository\\
$>$Being social\\
\href{https://help.github.com/articles/set-up-git/}{https://help.github.com/articles/set-up-git/}
\\
\\

\end{document}

%% Local Variables:
%% fill-column: 78
%% mode: auto-fill
%% compile-command: "make"
%% End:
