%% -*- Mode: LaTeX -*-
%%
%% template.tex
%% Created Wed Jul 25 09:32:40 AKDT 2007
%% by Raymond E. Marcil <marcilr@gmail.com>
%% 
%% Git is a free and open source distributed version control system designed
%% to handle everything from small to very large projects with speed and
%% efficiency.
%%
%% Git is easy to learn and has a tiny footprint with lightning fast
%% performance. It outclasses SCM tools like Subversion, CVS, Perforce,
%% and ClearCase with features like cheap local branching, convenient
%% staging areas, and multiple workflows.
%% https://git-scm.com/
%%

  %%
%%%%%% Preamble.
  %%

%% Specify DVIPS driver used by things like hyperref
\documentclass[12pt,letterpaper,dvips]{article}

%% rcs is the package to display cvs revision info.
%%\usepackage{rcs}
\usepackage{fullpage}
\usepackage{fancyvrb} 
\usepackage{graphicx}
\usepackage{figsize}
\usepackage{calc}

%%
%% enumitem – Control layout of itemize, enumerate, description
%% https://www.ctan.org/pkg/enumitem
%%
%% Allows for use of \bgein{itemize}[leftmargin=0pt] 
%% to lists with 0 left margin.
%%
%% Itemize left margin
%% http://tex.stackexchange.com/questions/170525/itemize-left-margin
%% 
\usepackage{enumitem}%     http://ctan.org/pkg/enumitem


%% caption package for use in justifying table or figure captions
\usepackage{caption}

\usepackage{xspace}
\usepackage{booktabs}
\usepackage[first,bottomafter]{draftcopy}
\usepackage[numbib]{tocbibind}

\usepackage{amssymb}              %% AMS Symbols, used for \checkmark
\usepackage{multicol}

%%
%% Extract SVN metadata for use elsewhere.
%% This information has:
%% o the filename
%% o the revision number
%% o the date and time of the last Subversion co command
%% o name of the user who has done the action
%%
\usepackage{svninfo}
\svnInfo $Id$

%%
%% Hyperref package for embedding URLs for clickable links in PDFs, 
%% also specify PDF attributes here.
%%
%% The pdfborder={0 0 0} is what ellimated the blue box around the url
%% displayed by \href{}{}.
%%
%% The command pdfborder={0 0 1} would display a box with thickness of 1 pt.
%%
%% Hypertext marks in LATEX: a manual for hyperref
%% by Sebastian Rahtz and Heiko Oberdiek - November 2012
%% http://ctan.org/pkg/hyperref 
%% http://mirror.hmc.edu/ctan/macros/latex/contrib/hyperref/doc/manual.html
%%
\usepackage[
colorlinks,
linkcolor=blue,
%%colorlinks=false,
hyperindex=false,
urlcolor=blue,
pdfborder={0 0 0},
pdfauthor={Raymond E. Marcil},
pdftitle={Git, Revision Control},
pdfcreator={ps2pdf},
pdfsubject={git, revision control},
pdfkeywords={git, revision control}
]{hyperref}


%%
%% Extract RCS metadata for use elsewhere.
%% Jason figured this out, very cool.
%%
%%\RCS $Revision: 1.53 $
%%\RCS $Date: 2006/06/26 21:04:55 $


  %%
%%%%%% Customization.
  %%

% On letter paper with 10pt font the Verbatim environment has 65 columns.
% With 12pt font the environment has 62 columns.  Exceeding this will exceed
% the frame and will look ugly.  YHBW.  HAND.
\RecustomVerbatimEnvironment{Verbatim}{Verbatim}{frame=single}

\renewenvironment{description}
                 {\list{}{\labelwidth 0pt \iteminden-\leftmargin
                          \let\labelsep\hsize
                          \let\makelabel\descriptionlabel}}
                 {\endlist}
\renewcommand*\descriptionlabel[1]{\hspace\labelsep\sffamily\bfseries #1}


  %%
%%%%%% Commands.
  %%

\newcommand{\FIXME}[1]{\textsf{[FIXME: #1]}}
\newcommand{\cmd}[1]{\texttt{#1}}


%% Squeeze space above/below captions
\setlength{\abovecaptionskip}{4pt}   % 0.5cm as an example
\setlength{\belowcaptionskip}{4pt}   % 0.5cm as an example


%% Tex really adds a lot of whitespace to itemized 
%% lists so define a new command itemize* with a 
%% lot less whitespace.  Found this in the British
%% Tex faq.
\newenvironment{itemize*}%
  {\begin{itemize}%
    \setlength{\itemsep}{0pt}%
    \setlength{\parsep}{0pt}}%
  {\end{itemize}}

  
%%
%% Tex really adds a lot of whitespace to itemized 
%% lists so define a new command itemize* with a 
%% lot less whitespace.  Found this in the British
%% Tex faq.
%%
%% Tue Jun 23 13:22:04 AKDT 2015
%% =============================
%% Added [leftmargin=0.0mm] to set the left margin=0
%% This requires use of the enumitem package:
%%   \usepackage{enumitem}%     http://ctan.org/pkg/enumitem
%%
%% Itemize left margin
%% http://tex.stackexchange.com/questions/170525/itemize-left-margin
%%
\newenvironment{itemizenoleft*}%
  {\begin{itemize}[leftmargin=15.0pt]%
    \setlength{\itemsep}{0pt}%
    \setlength{\parsep}{0pt}}%
  {\end{itemize}}
  

%%
%% Tex really adds a lot of whitespace to itemized 
%% lists so define a new command enumerate* with a 
%% lot less whitespace.  Created using itemize*
%% pattern.  
%%
  \newenvironment{enumerate*}%
  {\begin{enumerate}%
    \setlength{\itemsep}{0pt}%
    \setlength{\parsep}{0pt}}%
  {\end{enumerate}}


%%
%% Tex really adds a lot of whitespace to itemized 
%% lists so define a new command enumerate* with a 
%% lot less whitespace.  Created using itemize*
%% pattern.  
%%
%% Tue Jun 23 13:22:04 AKDT 2015
%% =============================
%% Added [leftmargin=0.0mm] to set the left margin=0
%% This requires use of the enumitem package:
%%   \usepackage{enumitem}%     http://ctan.org/pkg/enumitem
%%
%% Itemize left margin
%% http://tex.stackexchange.com/questions/170525/itemize-left-margin
%%
\newenvironment{enumeratenoleft*}%
  {\begin{enumerate}[leftmargin=0.0mm]%
    \setlength{\itemsep}{0pt}%
    \setlength{\parsep}{0pt}}%
  {\end{enumerate}}


%% Squeeze space
\renewcommand\floatpagefraction{.9}
\renewcommand\topfraction{.9}
\renewcommand\bottomfraction{.9}
\renewcommand\textfraction{.1}   
\setcounter{totalnumber}{50}
\setcounter{topnumber}{50}
\setcounter{bottomnumber}{50}


  %%
%%%%%% Document.
  %%

\title{Git\\
       Revision Control}

\author{Raymond E. Marcil\\
        \texttt{$<$marcilr@gmail.com$>$}
}

% Display subversion revision and date under author on 1st page.
\date{Revision \svnInfoRevision
      \hspace{2pt}
      (\svnInfoLongDate)}


% Set date to RCS revision date
%%\date{Revision \RCSRevision
%%      \hspace{2pt}
%%      (\RCSDate)}


%%
%% Display SVN (subversion) version data at top right of 1st page,
%% This may be preferable to underneath the author.
%%
%%\rhead{Revision \svnInfoRevision\\
%%\svnInfoLongDate}


\begin{document}

\maketitle

\begin{abstract}
  \noindent Git is a free and open source distributed version control
  system designed to handle everything from small to very large projects
  with speed and efficiency.\\
  \\
  Git is easy to learn and has a tiny footprint with lightning fast
  performance. It outclasses SCM tools like Subversion, CVS, Perforce,
  and ClearCase with features like cheap local branching, convenient
  staging areas, and multiple workflows.\footnote{Git - \href{https://git-scm.com/}{https://git-scm.com/}}
\end{abstract}

\vspace{2.0in}

%% Draw DNR logo and address at bottom of page
%%\begin{figure}[h]
%%        \hspace{0.32in}
%%        \SetFigLayout{1}{1}
%%        \begin{minipage}[b]{0.16\figwidth}
%%                \includegraphics[width=\textwidth]{dnr_bwlogo.eps}
%%        \end{minipage}
%%        \hspace{5pt}
%%        \begin{minipage}[b]{\figwidth}
%%                \bf{Alaska Department of Natural Resources}\\
%%                \small{\sf{Division of Support Services\\
%%                Land Records Information Section\\
%%                550 W. 7th Ave. Suite 706\\
%%                Anchorage, Alaska 99501}}
%%        \end{minipage}
%%\end{figure}

\newpage
\tableofcontents

\newpage
\listoffigures
\listoftables


%% =============== List of Abbreviations ===============
%% =============== List of Abbreviations ===============
\newpage
\setcounter{secnumdepth}{0}
\section{List of Definitions and Abbreviations}
\begin{itemize*}
  \item{\begin{bf}Branch\end{bf}} - \FIXME{Need data}
  \item{\begin{bf}Git\end{bf}} - Quoting Linus: ``I'm an egotistical bastard, and I name all my projects after myself. First 'Linux', now 'Git'''.\\
('git' is British slang for ``pig headed, think they are always correct,
    argumentative'').\footnote{Git FAQ\\
    \href{https://git.wiki.kernel.org/index.php/GitFaq\#Why\_the\_.27Git.27\_name.3F}{https://git.wiki.kernel.org/index.php/GitFaq\#Why\_the\_.27Git.27\_name.3F}}
  \item{\begin{bf}Tag\end{bf}} - \FIXME{Need data}
\end{itemize*}


%% ====================== Introduction ===========================
%% ====================== Introduction ===========================
\newpage
\section{Introduction}

Git is a distributed revision control system with an emphasis on
speed,\footnote{ Torvalds, Linus
  (2005-04-07). ``\href{http://marc.info/?l=linux-kernel&m=111288700902396}{Re:
    Kernel SCM saga...}'' linux-kernel (Mailing list). ``So I'm writing some
  scripts to try to track things a whole lot faster.''} data integrity,\footnote{ Torvalds, Linus (2007-06-10). ``\href{http://marc.info/?l=git&m=118143549107708}{Re: fatal: serious inflate inconsistency}''. git (Mailing list). A brief description of Git's data integrity design goals.} and support for distributed, non-linear workflows.\footnote{Linus Torvalds (2007-05-03). \href{https://www.youtube.com/watch?v=4XpnKHJAok8}{Google tech talk: Linus Torvalds on git}. Event occurs at 02:30. Retrieved 2007-05-16.} Git was initially designed and developed by Linus Torvalds for Linux kernel development in 2005, and has since become one of the most widely adopted version control systems for software development.\footnote{ ``\href{http://ianskerrett.wordpress.com/2014/06/23/eclipse-community-survey-2014-results/}{Eclipse Community Survey 2014 results | Ian Skerrett}''. Ianskerrett.wordpress.com. 2014-06-23. Retrieved 2014-06-23.}\\
\\
\noindent As with most other distributed revision control systems,
and unlike most client–server systems, every Git working directory is a
full-fledged repository with complete history and full version-tracking
capabilities, independent of network access or a central
server.\footnote{Chacon, Scott (24
  December 2014). \href{http://git-scm.com/book/en/v2}{Pro Git} (2nd ed.). New
  York, NY: Apress. pp. 29–30. ISBN 978-1484200773.} Like the Linux kernel,
Git is free software distributed under the terms of the GNU General Public
License version 2.\footnote{Git (software), From Wikipedia, the free
encyclopedia, \href{https://en.wikipedia.org/wiki/Git\_(software)}{https://en.wikipedia.org/wiki/Git\_(software)}}


%% ===================== Command Reference =======================
%% ===================== Command Reference =======================
\section{Command Reference}
\FIXME{Need data here.}


%% =================== Branching & Tagging =======================
%% =================== Branching & Tagging =======================
\newpage
\section{Branching \& Tagging}

In short: Best practice is branch out, merge often and keep
always in sync.\\
\\
\noindent There are pretty clear conventions about keeping your
code in a separate branches from master branch:

\begin{enumerate*}
\item You are about to make an implementation of major or
disruptive change
\item You are about to make some changes that might not be used
\item You want to experiment on something that you are not sure
    it will work
\item When you are told to branch out, others might have
    something they need to do in master
\end{enumerate*}

\noindent Rule of thumb is after branching out, you should keep in sync with
the master branch. Because eventually you need to merge it back to master. In
order to avoid a huge complicated mess of conflicts when merging back, you
should commit often, merge often.\footnote{Git branching and tagging best
practices\\
\href{http://programmers.stackexchange.com/questions/165725/git-branching-and-tagging-best-practices}{http://programmers.stackexchange.com/questions/165725/git-branching-and-tagging-best-practices}}


%% ====================== Project Host ===========================
%% ====================== Project Host ===========================
\newpage
\section{Project Host}

\subsection{Bitbucket vs. GitHub}


%% ========================== Repo ===============================
%% ========================== Repo ===============================
\newpage
\section{Repo}



%% ======================== Appendix =============================
%% ======================== Appendix =============================
%%
%% This will add a standard non-numbered Appendix to the document.
%% The next section Appendix chnages secnumdepth such that the Appendix
%% is not numbered but still displayed in the table of contents (TOC).
%%
%% Adding unnumbered sections to TOC
%% http://tex.stackexchange.com/questions/11668/adding-unnumbered-sections-to-toc
%% 
%% \section*{Appendix}
%% Need content here.
%%
\newpage
\setcounter{secnumdepth}{0}
\section{Appendix}


%% ============= Links ===============
A successful Git branching model\\
by Vincent Driessen on Tuesday, January 05, 2010\\
Fine branching diagram here.\\
\href{http://nvie.com/posts/a-successful-git-branching-model/}{http://nvie.com/posts/a-successful-git-branching-model/}
\\
\\
Bitbucket vs. GitHub: Which project host has the most?\\
The right choice boils down to a number of factors -- you might even consider using both\\
\href{http://www.infoworld.com/article/2611771/application-development/application-development-bitbucket-vs-github-which-project-host-has-the-most.html}{http://www.infoworld.com/article/2611771/application-development/application-development-bitbucket-vs-github-which-project-host-has-the-most.html}
\\
\\
Fetching a remote\\
$>$ \cmd{git clone}\\
$>$ \cmd{git fetch}\\
$>$ \cmd{git merge}\\
$>$ \cmd{git pull}\\
\href{https://help.github.com/articles/fetching-a-remote/}{https://help.github.com/articles/fetching-a-remote/}
\\
\\
Git\\
\href{https://git-scm.com/}{https://git-scm.com/}
\\
\\
Git (software)\\
From Wikipedia, the free encyclopedia\\
\href{https://en.wikipedia.org/wiki/Git\_(software)}{https://en.wikipedia.org/wiki/Git\_(software)}
\\
\\
Git About\\
\href{https://git-scm.com/about}{https://git-scm.com/about}
\\
\\
Git branching and tagging best practices\\
Excellent details and semantics.\\
\href{http://programmers.stackexchange.com/questions/165725/git-branching-and-tagging-best-practices}{http://programmers.stackexchange.com/questions/165725/git-branching-and-tagging-best-practices}
\\
\\
Git FAQ\\
\href{https://git.wiki.kernel.org/index.php/GitFaq}{https://git.wiki.kernel.org/index.php/GitFaq}
\\
\\
Pro Git (the git book)\\
Available as
  \href{https://progit2.s3.amazonaws.com/en/2015-08-09-23511/progit-en.661.pdf}{pdf},
  \href{https://progit2.s3.amazonaws.com/en/2015-08-09-23511/progit-en.661.epub}{epub},
  \href{https://progit2.s3.amazonaws.com/en/2015-08-09-23511/progit-en.661.mobi}{mobi},
  and \href{https://progit2.s3.amazonaws.com/en/2015-08-09-23511/progit-en.661.zip}{html}.\\
\href{http://git-scm.com/book/https://progit2.s3.amazonaws.com/en/2015-08-09-23511/progit-en.661.zipen/v2}{http://git-scm.com/book/en/v2}
\\
\\

\end{document}

%% Local Variables:
%% fill-column: 78
%% mode: auto-fill
%% compile-command: "make"
%% End:
