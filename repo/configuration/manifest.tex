%% -*- Mode: LaTeX -*-
%%
%% manifest.tex
%% Created Fri Nov 20 11:04:59 AKST 2015
%% Copyright (C) 2015 by Raymond E. Marcil <marcilr@gmail.com>
%%
%% Manifest
%%


%% ======================== Manifest =============================
%% ======================== Manifest =============================
\clearpage
\newpage
\subsubsection{Manifest}
The repo keeps a manifest, ``within the hidden directory named `.repo','' 
in ``a git project named `\cmd{manifests}' which
usually contains a file named `\cmd{default.xml}'.  This file
contains information about all the projects and where their associated
git repositories are located.  This file is also versioned thus when
you use the `\cmd{repo init -b XYZ}' command it will be reverted and you
can back to older branches that may have added/removed git projects
compared to the head.''\footnote{How does the Android repo manifest repository work?\\
\href{http://stackoverflow.com/questions/6149725/how-does-the-android-repo-manifest-repository-work}
{http://stackoverflow.com/questions/6149725/how-does-the-android-repo-manifest-repository-work}}
\\
\\
The \cmd{default.xml} file is symlinked to \cmd{.repo/manifest.xml} and
is created when the repo was initialized using:
\\
\\
\cmd{repo init -u $<$manifest path$>$}
\\
\\

%% ======================== Examples =============================
%% ======================== Examples =============================
\noindent \begin{bf}Examples\end{bf}
\\
\\
Following is a manifest, in \cmd{.repo/manifests/default.xml} file, showing use
of GitHub with username, ssh:// URL syntax, and 3 project repos
with different usernames:\footnote{Keiji Ariyama, \href{https://github.com/keiji/repo-sample/blob/master/default.xml}{https://github.com/keiji/repo-sample/blob/master/default.xml}}

\begin{Verbatim}
 <?xml version="1.0" encoding="UTF-8"?>
 <manifest>
   <remote name="origin" fetch="ssh://git@github.com/" />
   <default revision="master" remote="origin" />

   <project path="lib/plist-lib"
            name="keiji/AndroidPListLib.git" remote="origin" />

   <project path="lib/json-pull-parser"
            name="vvakame/JsonPullParser.git" remote="origin" />

   <project path="apps/twicca_megane_plugin"
            name="zaki50/TwiccaMeganePlugin.git" remote="origin" />
 </manifest>
\end{Verbatim}
