%% -*- Mode: LaTeX -*-
%%
%% commands.tex
%% Created Fri Nov 20 11:16:34 AKST 2015
%% Copyright (C) 2015 by Raymond E. Marcil <marcilr@gmail.com>
%%
%% Google Repo Commands
%%


%% ======================== Commands =============================
%% ======================== Commands =============================
\newpage
\subsection{Commands}
Repo usage takes the following form:\footnote{Repo command reference\\
\href{https://source.android.com/source/using-repo.html\#help}{https://source.android.com/source/using-repo.html\#help}}
\\
\indent \cmd{repo <COMMAND> <OPTIONS>}
\\
\\
Optional elements are shown in brackets [ ].  For example, many commands take
a project list as an argument.  You can specify project-list as a list of
names or a list of paths to local source directories for the projects:

\indent \cmd{repo sync [<PROJECT0> <PROJECT1> <PROJECTN>]}\\
\indent \cmd{repo sync [</PATH/TO/PROJECT0> ... </PATH/TO/PROJECTN>]}
\\

\noindent Once Repo is installed, you can find the latest documentation
starting with a summary of all commands by running:
\\
\indent \cmd{repo help}
\\
\\
\noindent You can get information about any command by running this within a Repo tree:
\\
\indent \cmd{repo help <COMMAND>}
\\
\\
NOTE: For \cmd{repo} commands without syntax here see the 
Repo command reference.\footnote{Ibid.}


\begin{table}[htb]
\begin{center}
\begin{tabular}{|p{.25\textwidth}|p{.47\textwidth}|}\hline
Command&Description\\
\hline
abandon&Permanently abandon a development branch\\
branch&View current topic branches\\
branches&View current topic branches\\
checkout&Checkout a branch for development\\
cherry-pick&Cherry-pick a change\\
diff&Show changes between commit and working tree\\
diffmanifests&Manifest diff utility\\
download&Download and checkout a change\\
grep&Print lines matching a pattern\\
forall&Executes the given shell command in each project.\footnote{Repo command reference --
\href{https://source.android.com/source/using-repo.html\#forall}{https://source.android.com/source/using-repo.html\#forall}}\\
help&Display detailed help on a command\\
info&Get info on the manifest branch, current branch or unmerged branches\\
init&Install repo in the current working directory\\
list&List projects and their associated directories\\
overview&Display overview of unmerged project branches\\
prune&Prune (delete) already merged topics\\
rebase&Rebase local branches on upstream branch\\
start& Start a new branch for development\\
status&Show the working tree status\\
sync&Update working tree to the latest revision\\
upload&Upload changes for code review\\
\hline
\end{tabular}
\caption {Repo Commands}
\label{table:commands}
\end{center}
\end{table}


%% Google repo init
%% -*- Mode: LaTeX -*-
%%
%% init.tex
%% Created Fri Nov 20 11:16:34 AKST 2015
%% Copyright (C) 2015 by Raymond E. Marcil <marcilr@gmail.com>
%%
%% Google repo init
%%

%% ========================== init ===============================
%% ========================== init ===============================
\clearpage
\subsubsection{init}
\FIXME{Good init syntax here:\\
\href{https://source.tizen.org/documentation/developer-guide/getting-started-guide/cloning-tizen-source}{https://source.tizen.org/documentation/developer-guide/getting-started-guide/cloning-tizen-source}}
\\
\\
\cmd{\$ repo init -u <URL> [<OPTIONS>]}
\\
\\
Installs Repo in the current directory. This creates a \cmd{.repo/}
directory that contains Git repositories for the Repo source code
and the standard Android manifest files.  The \cmd{.repo/}
directory also contains \cmd{manifest.xml}, which is a symlink to the
selected manifest in the \cmd{.repo/manifests/} directory.\footnote{Repo command reference --
\href{https://source.android.com/source/using-repo.html\#forall}{https://source.android.com/source/using-repo.html}}\\

\begin{table}[htb]
\begin{center}
\begin{tabular}{|p{.15\textwidth}|p{.75\textwidth}|}\hline
\centering Command&\centering Description\tabularnewline
\hline
\centering -u&Specify a URL from which to retrieve a manifest repository.
The common manifest can be found at:
\cmd{https://android.googlesource.com/platform/manifest}\\
\centering -m&Select a manifest file within the repository. If no manifest
 name is selected, the default is default.xml.\\
\centering -b&Specify a revision, i.e., a particular manifest-branch.\\
\hline
\end{tabular}
\caption {init Options}
\label{table:init_options}
\end{center}
\end{table}

%% ======================== Examples =============================
%% ======================== Examples =============================
\noindent \begin{bf}Examples\end{bf}
\\
\\
This will create a new place to hold your local copy of the source
tree.  The \cmd{url} should point to a Manifest repository that
describes the whole sources.  It is a special project with a file
(default.xml) that lists all the projects that Android is made of.
In the Manifest file, each projects has attributes about: where to
place it in the tree, where to download it from (git server),
revision that will be used (usually a branch name, tag or commit
sha-id).\footnote{Repo: Tips \& Tricks,\\
\href{http://xda-university.com/as-a-developer/repo-tips-tricks}{http://xda-university.com/as-a-developer/repo-tips-tricks}}
\\
\\
\indent\cmd{repo init -u <url> -b <branch>}
\\
\\
\noindent Note: For all remaining Repo commands, the current working
directory must either be the parent directory of \cmd{.repo/}
or a subdirectory of the parent directory.\footnote{Repo command reference --
\href{https://source.android.com/source/using-repo.html\#forall}{https://source.android.com/source/using-repo.html}}\\
\\
\\


%% ==================== GitHub repo init =========================
%% ==================== GitHub repo init =========================
\newpage
\noindent \begin{bf}GitHub repo init\end{bf}
\\
\\
\begin{verbatim}
 $ repo init -u https://github.com/marcilr/git.git -b <branch>
 $ repo sync
\end{verbatim}

\begin{verbatim}
covellite:~/tmp/foo$ repo init -u ssh://github.com/marcilr/.repo -m manifest.xml
Get https://gerrit.googlesource.com/git-repo/clone.bundle
Get https://gerrit.googlesource.com/git-repo
remote: Counting objects: 7, done
remote: Finding sources: 100% (182/182)
remote: Total 182 (delta 59), reused 182 (delta 59)
Receiving objects: 100% (182/182), 246.65 KiB | 0 bytes/s, done.
Resolving deltas: 100% (59/59), completed with 4 local objects.
From https://gerrit.googlesource.com/git-repo
   5d0c3a6..e1e0bd1  master     -> origin/master
   163fdbf..745b4ad  stable     -> origin/stable
 * [new tag]         v1.12.27   -> v1.12.27
 * [new tag]         v1.12.28   -> v1.12.28
 * [new tag]         v1.12.29   -> v1.12.29
 * [new tag]         v1.12.30   -> v1.12.30
 * [new tag]         v1.12.30.1 -> v1.12.30.1
 * [new tag]         v1.12.31   -> v1.12.31
 * [new tag]         v1.12.32   -> v1.12.32
Get ssh://github.com/marcilr/.repo
Permission denied (publickey).
Permission denied (publickey).
fatal: Could not read from remote repository.

Please make sure you have the correct access rights
and the repository exists.
Permission denied (publickey).
fatal: Could not read from remote repository.

Please make sure you have the correct access rights
and the repository exists.
fatal: cannot obtain manifest ssh://github.com/marcilr/.repo
covellite:~/tmp/foo$
\end{verbatim}