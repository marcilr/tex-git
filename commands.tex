%% -*- Mode: LaTeX -*-
%%
%% commands.tex
%% Created Fri Nov 20 11:16:34 AKST 2015
%% Copyright (C) 2015 by Raymond E. Marcil <marcilr@gmail.com>
%%
%% Google Repo Commands
%%


%% ======================== Commands =============================
%% ======================== Commands =============================
\newpage
\subsection{Commands}
Repo usage takes the following form:\footnote{Repo command reference\\
\href{https://source.android.com/source/using-repo.html\#help}{https://source.android.com/source/using-repo.html\#help}}
\\
\indent \cmd{repo <COMMAND> <OPTIONS>}
\\
\\
Optional elements are shown in brackets [ ].  For example, many commands take
a project list as an argument.  You can specify project-list as a list of
names or a list of paths to local source directories for the projects:

\indent \cmd{repo sync [<PROJECT0> <PROJECT1> <PROJECTN>]}\\
\indent \cmd{repo sync [</PATH/TO/PROJECT0> ... </PATH/TO/PROJECTN>]}
\\

\noindent Once Repo is installed, you can find the latest documentation
starting with a summary of all commands by running:
\\
\indent \cmd{repo help}
\\
\\
\noindent You can get information about any command by running this within a Repo tree:
\\
\indent \cmd{repo help <COMMAND>}
\\
\\
NOTE: For \cmd{repo} commands without syntax here see the 
Repo command reference.\footnote{Ibid.}


\begin{table}[htb]
\begin{center}
\begin{tabular}{|p{.25\textwidth}|p{.47\textwidth}|}\hline
Command&Description\\
\hline
abandon&Permanently abandon a development branch\\
branch&View current topic branches\\
branches&View current topic branches\\
checkout&Checkout a branch for development\\
cherry-pick&Cherry-pick a change\\
diff&Show changes between commit and working tree\\
diffmanifests&Manifest diff utility\\
download&Download and checkout a change\\
grep&Print lines matching a pattern\\
forall&Executes the given shell command in each project.\footnote{Repo command reference --
\href{https://source.android.com/source/using-repo.html\#forall}{https://source.android.com/source/using-repo.html\#forall}}\\
help&Display detailed help on a command\\
info&Get info on the manifest branch, current branch or unmerged branches\\
init&Install repo in the current working directory\\
list&List projects and their associated directories\\
overview&Display overview of unmerged project branches\\
prune&Prune (delete) already merged topics\\
rebase&Rebase local branches on upstream branch\\
start& Start a new branch for development\\
status&Show the working tree status\\
sync&Update working tree to the latest revision\\
upload&Upload changes for code review\\
\hline
\end{tabular}
\caption {Repo Commands}
\label{table:commands}
\end{center}
\end{table}
