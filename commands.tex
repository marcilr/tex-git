%% -*- Mode: LaTeX -*-
%%
%% commands.tex
%% Created Thu Nov 19 08:25:10 AKST 2015
%% Copyright (C) 2015 by Raymond E. Marcil <marcilr@gmail.com>
%%
%% Remote Commands
%%

%% ===================== Remote Commands =========================
%% ===================== Remote Commands =========================
\newpage
\subsubsection{Remote Commands}
\begin{table}[htb]
\begin{center}
\begin{tabular}{|p{.20\textwidth}|p{.70\textwidth}|}\hline
Command&Description\\
\hline
\cmd{remote add}&Add a new remote in the directory your repository is stored at.\\
\cmd{remote set-url}&Change a remote's URL.\\
\cmd{remote rename}&Rename an existing remote.\\
\cmd{remote rm}&Remove a remote URL from your repository.\\
\hline
\end{tabular}
\caption {Remote Commands}
\label{table:remote_commands}
\end{center}
\end{table}

\noindent For further reading see ``\href{http://git-scm.com/book/en/Git-Basics-Working-with-Remotes}{Working with Remotes}''
from the Pro Git book.\footnote{Pro Git book, \href{http://git-scm.com/book/en/Git-Basics-Working-with-Remotes}{http://git-scm.com/book/en/Git-Basics-Working-with-Remotes}}


%% git remote add
%% -*- Mode: LaTeX -*-
%%
%% git-remote-add.tex
%% Created Thu Nov 19 08:25:10 AKST 2015
%% Copyright (C) 2015 by Raymond E. Marcil <marcilr@gmail.com>
%%
%% git remote add
%%


%% ===================== git remote add =========================
%% ===================== git remote add =========================
\newpage
\subsubsection{git remote add}
To add a new remote, use the \cmd{git remote add} command on the
terminal, in the directory your repository is stored at.\footnote{Adding a remote,
\href{https://help.github.com/articles/adding-a-remote/}{https://help.github.com/articles/adding-a-remote/}}
\\
\\
The \cmd{git remote add} command takes two arguments:
\vspace{-10pt}
\begin{verbatim}
$ git remote add <NAME> <REMOTE_URL> 
\end{verbatim}

\noindent Where:
\vspace{-10pt}
\begin{table}[htb]
%%\begin{table}
\begin{center}
\begin{tabular}{p{.20\textwidth}p{.80\textwidth}}
\hspace{20pt}\cmd{<NAME>}&- A remote name, for example, \cmd{origin}\\
\hspace{20pt}\cmd{<REMOTE\_URL>}&- A remote URL, for example, \cmd{https://github.com/user/repo.git}\\
\end{tabular}
\label{table:remote_add}
\end{center}
\end{table}

\vspace{-20pt}
\noindent Git associates a remote URL with a name, and your
default remote is usually called origin.\footnote{About remote repositories,
\href{https://help.github.com/articles/about-remote-repositories/}{https://help.github.com/articles/about-remote-repositories/}}
\\

%% ======================== Examples ============================
%% ======================== Examples ============================
\vspace{10pt}
\noindent \begin{bf}Examples\end{bf}
\begin{Verbatim}
 # This associates the name origin with SSH URL for repo.git repository.
 $ git remote add origin git@github.com:user/repo.git
\end{Verbatim}

\noindent Alternatively using https syntax:

\begin{Verbatim}
 $ git remote add origin https://github.com/user/repo.git
 # Set a new remote

 $ git remote -v
 # Verify new remote
 origin  https://github.com/user/repo.git (fetch)
 origin  https://github.com/user/repo.git (push)
\end{Verbatim}



%% =================== git remote set-url =======================
%% =================== git remote set-url =======================
\newpage
\subsubsection{git remote set-url}
The \cmd{git remote set-url} command changes an existing remote
repository URL.\footnote{Changing a remote's URL,
\href{https://help.github.com/articles/changing-a-remote-s-url/}{https://help.github.com/articles/changing-a-remote-s-url/}}
\\
\\
The git remote set-url command takes two arguments:
\\
$>$ An existing remote name. For example, \cmd{origin} or \cmd{upstream} are two common choices.\\
$>$ A new URL for the remote. For example:\\

\noindent\hspace*{10pt}$>$ If you're updating to use HTTPS, your URL might look like:\\
\noindent\hspace*{23pt}\cmd{https://github.com/USERNAME/OTHERREPOSITORY.git}\\
\\
\noindent\hspace*{10pt}$>$ If you're updating to use SSH, your URL might look like:\\
\noindent\hspace*{23pt}\cmd{git@github.com:USERNAME/OTHERREPOSITORY.git}


%% ========= Switching remote URLs from SSH to HTTPS ===========
%% ========= Switching remote URLs from SSH to HTTPS ===========
\subsubsection{Switching remote URLs from SSH to HTTPS}
\begin{enumerate}
  \item{Open Terminal (for Mac and Linux users) or the command prompt
       (for Windows users).\footnote{Switching remote URLs from HTTPS to SSH,
       \href{https://help.github.com/articles/changing-a-remote-s-url/}{https://help.github.com/articles/changing-a-remote-s-url/}}}
  \item{Change the current working directory to your local project.}
  \item{List your existing remotes in order to get the name of the remote you want to change.
\begin{Verbatim}
 $ git remote -v
 # origin  git@github.com:USERNAME/REPOSITORY.git (fetch)
 # origin  git@github.com:USERNAME/REPOSITORY.git (push)
\end{Verbatim}
       }
  \item{Change your remote's URL from SSH to HTTPS with the git remote set-url command.
\begin{Verbatim}
 $ git remote set-url origin \
 https://github.com/USERNAME/OTHERREPOSITORY.git
\end{Verbatim}
      }

  \item{Verify that the remote URL has changed.
\begin{Verbatim}
$ git remote -v
# Verify new remote URL
# origin  https://github.com/USERNAME/OTHERREPOSITORY.git (fetch)
# origin  https://github.com/USERNAME/OTHERREPOSITORY.git (push)
\end{Verbatim}
       }
\end{enumerate}


%% ========= Switching remote URLs from HTTPS to SSH ===========
%% ========= Switching remote URLs from HTTPS to SSH ===========
\newpage
\subsubsection{Switching remote URLs from HTTPS to SSH}
\begin{enumerate}
  \item{Open Terminal (for Mac and Linux users) or the command prompt
       (for Windows users).\footnote{Switching remote URLs from HTTPS to SSH,
       \href{https://help.github.com/articles/changing-a-remote-s-url/}{https://help.github.com/articles/changing-a-remote-s-url/}}}
  \item{Change the current working directory to your local project.}
  \item{List your existing remotes in order to get the name of the remote you want to change.
\begin{Verbatim}
 $ git remote -v
 origin  https://github.com/USERNAME/REPOSITORY.git (fetch)
 origin  https://github.com/USERNAME/REPOSITORY.git (push)
\end{Verbatim}
       }
  \item{Change your remote's URL from HTTPS to SSH with the git remote set-url command.
\begin{Verbatim}
 $ git remote set-url origin \
 git@github.com:USERNAME/OTHERREPOSITORY.git
\end{Verbatim}
      }

  \item{Verify that the remote URL has changed.
\begin{Verbatim}
 $ git remote -v
 # Verify new remote URL
 origin  git@github.com:USERNAME/OTHERREPOSITORY.git (fetch)
 origin  git@github.com:USERNAME/OTHERREPOSITORY.git (push)
\end{Verbatim}
       }
\end{enumerate}


%% ==================== git remote rename ======================
%% ==================== git remote rename ======================
\subsubsection{git remote rename}
Use the git remote rename command to rename an existing remote.\footnote{Renaming a remote,
\href{https://help.github.com/articles/renaming-a-remote/}{https://help.github.com/articles/renaming-a-remote/}}
\\
\\
The git remote rename command takes two arguments:
\\
\\
$>$ An existing remote name, for example, origin A new name for the\\
$>$ remote, for example, destination

\newpage
\noindent \begin{bf}Example\end{bf}\\
The examples below assume you're cloning using HTTPS, which is recommended.

\begin{Verbatim}
 $ git remote -v
 # View existing remotes
 origin  https://github.com/OWNER/REPOSITORY.git (fetch)
 origin  https://github.com/OWNER/REPOSITORY.git (push)

 $ git remote rename origin destination
 # Change remote name from 'origin' to 'destination'

 $ git remote -v
 # Verify remote's new name
 destination  https://github.com/OWNER/REPOSITORY.git (fetch)
 destination  https://github.com/OWNER/REPOSITORY.git (push)
\end{Verbatim}


%% ====================== git remote rm ======================
%% ====================== git remote rm ======================
\subsubsection{git remote rm}
Use the \cmd{git remote rm} command to remove a remote URL
from your repository.\footnote{Removing a remote,
\href{https://help.github.com/articles/removing-a-remote/}{https://help.github.com/articles/removing-a-remote/}}
\\
\\
The \cmd{git remote rm} command takes one argument:
\\
\\
$>$ A remote name, for example, destination
\\
\\
\noindent \begin{bf}Example\end{bf}
\\
\\
The examples below assume you're cloning using HTTPS, which is recommended.

\begin{Verbatim}
 $ git remote -v
 # View current remotes
 origin  https://github.com/OWNER/REPOSITORY.git (fetch)
 origin  https://github.com/OWNER/REPOSITORY.git (push)
 destination  https://github.com/FORKER/REPOSITORY.git (fetch)
 destination  https://github.com/FORKER/REPOSITORY.git (push)

 $ git remote rm destination
 # Remove remote
 $ git remote -v
 # Verify it's gone
 origin  https://github.com/OWNER/REPOSITORY.git (fetch)
 origin  https://github.com/OWNER/REPOSITORY.git (push)
\end{Verbatim}

\noindent Note: \cmd{git remote rm} does not delete the remote
repository from the server.  It simply removes the remote and
its references from your local repository.
