%% -*- Mode: LaTeX -*-
%%
%% clone.tex
%% Created Wed Nov 18 15:00:41 AKST 2015
%% Copyright (C) 2015 by Raymond E. Marcil <marcilr@gmail.com>
%%
%% Clone
%%

%% =========================== Clone =============================
%% =========================== Clone =============================
\newpage
\subsubsection{Clone}
To grab a complete copy of another user's repository, use git clone like this:

\begin{Verbatim}
  $ git clone https://github.com/USERNAME/REPOSITORY.git
  # Clones a repository to your computer
\end{Verbatim}

\noindent When you run \cmd{git clone}, the following actions occur:\\
$>$ A new folder called repo is made\\
$>$ It is initialized as a Git repository\\
$>$ A remote named origin is created, pointing to the URL you cloned from\\
$>$ All of the repository's files and commits are downloaded there\\
$>$ The default branch (usually called master) is checked out\\

\noindent For every branch \cmd{foo} in the remote repository, a corresponding remote-tracking
branch \cmd{refs/remotes/origin/foo} is created in your local repository.  You can
usually abbreviate such remote-tracking branch names to \cmd{origin/foo}.\footnote{Fetching a remote,
\cmd{git clone}, \cmd{git fetch}, \cmd{git merge}, \cmd{git pull},
\href{https://help.github.com/articles/fetching-a-remote/}{https://help.github.com/articles/fetching-a-remote/}}


%% ========================= Examples ============================
%% ========================= Examples ============================
\subsubsection{Examples}
To clone repository named \cmd{git} from GitHub to local \cmd{covellite} workstation:
\begin{Verbatim}
 covellite:~$ git clone https://github.com/marcilr/git.git
 Cloning into 'git'...
 warning: You appear to have cloned an empty repository.
 Checking connectivity... done.
 covellite:~$
\end{Verbatim}

\noindent Clone \cmd{.repo} repository into \cmd{git/} directory:
\begin{Verbatim}
 covellite:~/git$ git clone https://github.com/marcilr/.repo
 Cloning into '.repo'...
 warning: You appear to have cloned an empty repository.
 Checking connectivity... done.
 covellite:~/git$
\end{Verbatim}


\newpage
\noindent To clone a Git repository over SSH, you can specify ssh:// URL like this:\\
\begin{Verbatim}
 $ git clone ssh://user@server/project.git
\end{Verbatim}

\noindent Or you can use the shorter scp-like syntax for the SSH protocol:

\begin{Verbatim}
 $ git clone user@server:project.git
\end{Verbatim}

\noindent You can also not specify a user, and Git assumes the user
you're currently logged in as.\footnote{
Git on the Server - The Protocols, The SSH Protocol,\\
\href{https://git-scm.com/book/en/v2/Git-on-the-Server-The-Protocols}{https://git-scm.com/book/en/v2/Git-on-the-Server-The-Protocols}}
\\
\\
\noindent \FIXME{Need more commands here.}
