%% -*- Mode: LaTeX -*-
%%
%% github.tex
%% Created Tue Nov 17 15:36:52 AKST 2015
%% Copyright (C) 2015 by Raymond E. Marcil <marcilr@gmail.com>
%%
%% GitHub
%%

%% ========================= GitHub ==============================
%% ========================= GitHub ==============================
\newpage
\subsubsection{GitHub}

\FIXME{Still need cli list, rename, and delete functionality.}
\\
\\
\noindent GitHub's collaborative approach to development depends on
publishing commits from your local repository for other people to
view, fetch, and update.\footnote{About remote repositories,
\href{https://help.github.com/articles/about-remote-repositories/}{https://help.github.com/articles/about-remote-repositories/}}
\\
\\

\begin{center}
\framebox{
\begin{minipage}[t]{0.9\textwidth}
\begin{bf}Pushing from local repository to GitHub hosted remote\end{bf}
\\
\\
You push your local repository to the remote repository
using the git push command after first establishing a
relationship between the two with the
\cmd{git remote add [alias] [url]} command. If you visit
your Github repository, it will show you the URL to use
for pushing. You'll first enter something like:\\
\hspace*{20pt}\cmd{git remote add origin git@github.com:username/reponame.git}
\\
\\
Unless you started by running git clone against the remote
repository, in which case this step has been done for you 
already.
\\
\\
And after that, you'll type:\\
\hspace*{20pt}\cmd{git push origin master}
\\
\\
After your first push, you can simply type:\\
\hspace*{20pt}\cmd{git push}
\\
\\
when you want to update the remote repository in the future.\\
\hspace*{20pt}edited Jul 10 '14 at 14:17 by thomio\\
\hspace*{20pt}answered May 13 '12 at 18:01 by larsks
\\
...
\\
Subversion implicitly has the remote repository associated with
it at all times. Git, on the other hand, allows many ``remotes'',
each of which represents a single remote place you can push to
or pull from.
\\
\\
You need to add a remote for the GitHub repository to your local
repository, then use \cmd{git push <REMOTE>} or
\cmd{git pull <REMOTE>} to push and pull respectively -
or the GUI equivalents.
\\
...
\\
Once you have associated the two you will be able to push
or pull branches.
\hspace*{20pt}answered May 13 '12 at 18:01\\
\hspace*{20pt}by Daniel Pittman
\end{minipage}
}
\end{center}

\footnotetext{Pushing from local repository to GitHub hosted remote,\\
\href{http://stackoverflow.com/questions/10573957/pushing-from-local-repository-to-github-hosted-remote}{http://stackoverflow.com/questions/10573957/pushing-from-local-repository-to-github-hosted-remote}}

%% ========= Caching your GitHub password in Git =================
%% ========= Caching your GitHub password in Git =================
\subsubsection{Caching your GitHub password in Git}
If you're cloning GitHub repositories using HTTPS, you can use a
credential helper to tell Git to remember your GitHub username
and password every time it talks to GitHub.
\\
\\
If you clone GitHub repositories using SSH, then you authenticate
using SSH keys instead of a username and password.  For help
setting up an SSH connection, see Generating SSH Keys.\footnote{Generating SSH keys,
\href{https://help.github.com/articles/generating-ssh-keys/}{https://help.github.com/articles/generating-ssh-keys/}}
\\
\\
Turn on the credential helper so that Git will save your password
in memory for some time.  By default, Git will cache your password
for 15 minutes.\footnote{Caching your GitHub password in Git,
\href{https://help.github.com/articles/caching-your-github-password-in-git/}{https://help.github.com/articles/caching-your-github-password-in-git/}}

\begin{enumerate}
  \item{In Terminal, enter the following:
\begin{Verbatim}
 $ git config --global credential.helper cache
 # Set git to use the credential memory cache
\end{Verbatim}
       }
  \item{To change the default password cache timeout, enter the following:
\begin{Verbatim}
 $ git config --global credential.helper 'cache --timeout=3600'
 # Set the cache to timeout after 1 hour (setting is in seconds)
\end{Verbatim}
       }
\end{enumerate}

\noindent \FIXME{The GitHub user and pass got saved by alternate means.  How was that done?}

\subsubsection{Create a Github Repo from the Command Line}
Creating a GitHub repository from the command line is incredibly
convenient.
\\
\\
Googled up some simple shell script to create GitHub repo via
command line:

\begin{verbatim}
  "curl -u $username:$token" https://api.github.com/user/repos \
  -d '{"name":"'$repo_name'"}'}
\end{verbatim}

\noindent To use, you could simply replace \cmd{\$username} with your
GitHub username, \cmd{\$token} with a Personal Access Token\footnote{GitHub
supports Personal access tokens, under Settings, click Personal access
tokens. Personal access tokens function like ordinary OAuth access tokens.
They can be used instead of a password for Git over HTTPS, or can be used
to authenticate to the API over Basic Authentication.  I set mine to the usual:)\\
\href{https://github.com/settings/tokens}{https://github.com/settings/tokens}}
for the same user (available for generation in your GitHub\\
Settings $>$ Applications), and \cmd{\$repo\_name} with your
desired new Repository name.\footnote{Create a Github Repo
from the Command Line, by Eli Fatsi - Jan 29, 2014,\\
\href{https://viget.com/extend/create-a-github-repo-from-the-command-line}{https://viget.com/extend/create-a-github-repo-from-the-command-line}}
\\
\\
Creating a repo from the command line is definitely faster
than going to Github and using the web app to get the job
done, but in order to truly make this task speedy, we need
some Bash programming.

\newpage
\begin{Verbatim}
github-create() {
  repo_name=$1
 
  dir_name=`basename $(pwd)`
 
  if [ "$repo_name" = "" ]; then
    echo "Repo name (hit enter to use '$dir_name')?"
    read repo_name
  fi
 
  if [ "$repo_name" = "" ]; then
    repo_name=$dir_name
  fi
 
  username=`git config github.user`
  if [ "$username" = "" ]; then
    echo "Could not find username, run 'git config \
    --global github.user <username>'"
    invalid_credentials=1
  fi
 
  token=`git config github.token`
  if [ "$token" = "" ]; then
    echo "Could not find token, run 'git config \
    --global github.token <token>'"
    invalid_credentials=1
  fi
 
  if [ "$invalid_credentials" == "1" ]; then
    return 1
  fi
 
  echo -n "Creating Github repository '$repo_name' ..."
  curl -u "$username:$token" https://api.github.com/user/repos \
  -d '{"name":"'$repo_name'"}' > /dev/null 2>&1
  echo " done."
 
  echo -n "Pushing local code to remote ..."
  git remote add origin git@github.com:$username/$repo_name.git \
  > /dev/null 2>&1
  git push -u origin master > /dev/null 2>&1
  echo " done."
}
\end{Verbatim}

\noindent Plop this function into your \cmd{\textasciitilde/.bash\_profile}, open a new
Terminal window or source \cmd{\textasciitilde/.bash\_profile}, and the function
will be loaded up and ready for use.
\\
\\
Then while in an existing git project, running \cmd{github-create}
will create the repo and push your master branch up in one shot.
You will need to set some github config variables (instructions
will be spit out if you don't have them).  Here’s an example:
\\
\\
\begin{Verbatim}
 BASH:projects $ rails new my_new_project
   ..... (a whole lot of generated things)
 BASH:projects $ cd my_new_project/
 BASH:my_new_project $ git init && git add . && git commit \
 -m 'Initial commit'
  ..... (a whole lot of git additions)
 BASH:my_new_project $ github-create
   Repo name (hit enter to use 'my_new_project')?
 
   Creating Github repository 'my_new_project' ... done.
   Pushing local code to remote ... done.
\end{Verbatim}

\noindent Had I called the function with an argument — \cmd{github-create my\_project} —
then it would have used the argument and skipped the Repo name question.\footnote{Create a Github Repo
from the Command Line, by Eli Fatsi - Jan 29, 2014,\\
\href{https://viget.com/extend/create-a-github-repo-from-the-command-line}{https://viget.com/extend/create-a-github-repo-from-the-command-line}}
\\
\\
On GCI Network Services, OSS \cmd{covellite} Debian jessie 8.2
workstation the \cmd{gtihub-create} did not execute.  Put
github-create() into a standalone \cmd{\textasciitilde/gtihub-create}
script with call to \cmd{gtihub-create()} under main.
\\
\\
Tested with:
\begin{Verbatim}
 $ mkdir ~/quux
 $ cd ~/quux
 $ git init
 Initialized empty Git repository in /home/marcilr/quux/.git/
 $ github-create 
 Repo name (hit enter to use 'quux')?

 Could not find username, run 'git config --global github.user <username>'
 Could not find token, run 'git config --global github.token <token>'
 $
\end{Verbatim}

\newpage
\noindent Configured the GetHub username and token to alleviate GutHub credential errors:
\begin{Verbatim}
 $ git config --global github.user marcilr
 $ git config --global github.token <token>
\end{Verbatim}

\noindent Was then able to run \cmd{\textasciitilde/github-create} successfully:
\begin{Verbatim}
 $ cd ~/quux/
 $ github-create
 Repo name (hit enter to use 'quux')? <enter>

 Creating Github repository 'quux' ... done.
 Pushing local code to remote ... done.
 $
\end{Verbatim}

\noindent Checking GitHub via online access I found the new \cmd{quux} repo.
\\
\\
\noindent \FIXME{Need Bitbucket vs. GitHub section}


%% Adding an existing project to GitHub using the command line
%% https://help.github.com/articles/adding-an-existing-project-to-github-using-the-command-line/

%% Adding a file to a repository from the command line
%% https://help.github.com/articles/adding-a-file-to-a-repository-from-the-command-line/
